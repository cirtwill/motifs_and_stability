\documentclass[12pt]{letter}
\usepackage{graphicx}
% \usepackage[britdate]{SU-letter}
\usepackage[britdate]{UH-letter}
\usepackage{times}
\usepackage{letterbib}
\usepackage{geometry}
\usepackage[round]{natbib}
\usepackage{graphicx}
\geometry{a4paper}
\usepackage[T1]{fontenc}
\usepackage[utf8]{inputenc}
\usepackage{authblk}
\usepackage[running]{lineno}
\usepackage{amsmath,amsfonts,amssymb}
% \usepackage[margin=10pt,font=small,labelfont=bf]{caption}

%\usepackage{natbib}
% \bibpunct[; ]{(}{)}{;}{a}{,}{;}

\newenvironment{refquote}{\bigskip \begin{it}}{\end{it}\smallskip}

\newenvironment{figure}{}

\position{Postdoctoral fellow}
\department{Department of Agricultural Sciences}
\location{University of Helsinki}
\cityzip{Helsinki, Finland}
\telephone{}
\fax{}
\email{alyssa.cirtwill@helsinki.fi}
\url{http://cirtwill.github.io}
\name{Dr. Alyssa R. Cirtwill}

\begin{document}

\begin{letter}{\bf Prof. Dries Bonte\\
Oikos Editorial Office \\
Lund University \\
Lund, Sweden}


\opening{Dear Prof. Bonte:}


    \closing{Best regards,}

    \clearpage

% % R1 preface

    % Thank you for the time and effort you have put into reviewing our manuscript, and for the helpful feedback provided. 
    % In light of this feedback, we have made a number of major changes to the manuscript. 
    % Following suggestions from both reviewers, we have changed our statistical methods to rely on simpler linear models and have removed the PLS regression. 
    % We now also look at \emph{where} in a motif species are (their motif position) as opposed to simply which motifs they appear in, in order to delve more deeply into how this may affect their stability. 
    % Finally, we focus on one version of motif roles (species-normalised) in the main text and report results for the other versions in the SI.
    % Because of these changes, which we believe these changes have substantially strengthened the manuscript, the results section is almost completely different from the previous version.
    % We have also substantially simplified the methods and this section should now be much clearer. 


    % We must also apologise for the long delay in returning this manuscript. It has been a particularly busy and difficult time for a number of reasons (largely Covid related), including one author moving continents. We also have to confess to having missed the statement of the deadline, so did not ask for an extension of time. We have submitted the revision as a new submission but include this reply and a tracked-changes document.

    
    % Below is a point-by-point response to the reviewers comments. Our responses  are preceded by a bold \textbf{R}. Sections quoted from the manuscript are indented.


% % Initial coverletter

%   My co-author and I are happy to invite you to consider our manuscript
%   \emph{Stable motifs delay species loss in simulated food webs} for publication in \textbf{Oikos}.


%   Motifs --unique configurations of small numbers of interacting species-- are becoming more common as a tool for describing the meso-scale structure of ecological networks (i.e., a species 'neighborhood' within the network, including direct and close indirect interactions). 
%   The set of motifs a species participates in has already been linked to species' traits, taxonomy, and location.
%   More recent research has also suggested that some 'stable' motifs, in isolation, dampen disturbances and facilitate persistence of the species within them while others do not.
%   This suggests that species which participate in many of these `stable' motifs should be more likely to persist following a disturbance than species which participate in more of the `unstable' motifs.
%   If this is the case, motifs offer the possibility of estimating extinction risk within all types of ecological communities, without necessarily having to know the structure of the entire network.


%   Here we test this possibility using a large set of realistic simulated networks. In brief, we find that participating in more of the stable motifs is generally associated with longer persistence. Participating more often in the omnivory motif, however, is associated with shorter persistence. Motifs can therefore be added to the toolbox of researchers wishing to estimate species' risk of extinction. In particular, motifs offer a more detailed perspective on network structure than simpler measures such as degree (number of interaction partners) or trophic level and may be useful when comparing the risk of apparently similar taxa.


%   Our results also offer a potential explanation for the variable effects of omnivory on community stability. In our networks, participating more often in the omnivory motif was associated with greater extinction risk. However, species with high numbers of interaction partners (which tend to have lower extinction risk) appeared more often in the omnivory motif than species with few interaction partners. The apparent effect of omnivory may therefore depend on the balance between the effect of omnivory \emph{per se} and the effect of numbers of interaction partners.


%   Predicting which species are most likely to go extinct following a disturbance is one of the pressing ecological tasks of our time. 
%   Motifs can contribute to solving this problem as they provide a rich description of the ways species fit into their communities which can be applied to any ecological community and, unlike whole-network measures of stability, can be calculated for a focal species or group even if the structure of an entire community is unknown.
%   Moreover, our results advance the basic science of network ecology by establishing the relationships between motif participation, degree, and trophic level. 
%   Because of the broad applicability and combination of theoretical and practical relevance of our results, we believe that \textbf{Oikos} is the best place to publish this manuscript. 
%   We hope that you will agree and eagerly await your reply.


%   A non-peer reviewed preprint of this manuscript has been posted to bioRxiv \\\noindent(doi: https://www.biorxiv.org/content/10.1101/2021.04.06.438635).

% \closing{Best regards,}


% Please highlight the changes to your manuscript within the document by using the track changes mode in MS Word or by using bold or colored text. Once the revised manuscript is prepared, you can upload it and submit it through your Author Center.


% PLEASE: Submit your revised manuscript as MS Word or OpenOffice file(s). Submit your manuscript as one complete file, which includes: the text, legends, illustrations, tables and references.

% Also submit, as separate files, all illustrations in high resolution (see above).


% Sincerely,

% Prof. Dries Bonte
% Editor-in-Chief, Oikos

{\large \textbf{Editor-in-Chief comments}}

Dear Dr. Cirtwill:

Manuscript ID OIK-09436 entitled "Stable motifs delay species loss in simulated food webs" which you submitted to Oikos, has been reviewed.  The comments of the reviewer(s) and recommendation by the SE are included at the bottom of this letter.


Author Contribution Indication
The contributions of each author to this work must now be indicated when you submit your revised manuscript. To add Author Contributions using CRediT taxonomy (http://credit.niso.org/contributor-roles-defined/), simply click the “Provide CRediT Contribution” link for each author in the ‘Authors \& Institutions’ step of the submission process. From there, you will be able to check applicable Author/Contributor Roles and, if available, specify the Degree of Contribution. You MUST provide this information as part of the revision process. Author Contributions will be published with the accepted article and cannot be edited after article acceptance. Therefore you must ensure the Author Contribution information you provide is accurate prior to final acceptance.


Please highlight the changes to your manuscript within the document by using the track changes mode in MS Word or by using bold or colored text. Once the revised manuscript is prepared, you can upload it and submit it through your Author Center.


TEXT: Please read the instruction for authors thoroughly before resubmitting your manuscript.

FIGURES: Please ensure that your illustrations are of high quality (> 300 dpi), preferably as high resolution jpeg, pdf or gifs. Please avoid pixelated symbols and legends. The normal width of an illustration in Oikos is 7.5 cm which means that all legends and symbols should be easily readable at this size. When scientifically motivated, a larger width, 11 cm, is also available.

REFERENCES: Please revise the reference list so that it complies with our style as seen in current issues of Oikos. Papers with incomplete or malformatted reference lists, will not be officially accepted for publication.

MANUSCRIPT DETAILS: If any changes have been made to the title and/or abstract, please ensure that the relevant parts in the manuscript details within your Author Center are updated accordingly.

PLEASE: Submit your revised manuscript as MS Word or OpenOffice file(s). Submit your manuscript as one complete file, which includes: the text, legends, illustrations, tables and references.

Also submit, as separate files, all illustrations in high resolution (see above).

Once again, thank you for submitting your manuscript to Oikos and I look forward to receiving your revision.

Sincerely,

Prof. Dries Bonte
Editor-in-Chief, Oikos


\clearpage

{\large \textbf{Reply to Subject Editor}}

  \begin{quotation}
    Recommendation by the Subject Editor (Dr. Fran\c{c}ois Munoz):

    We have received three insightful reviews of this manuscript.

    Based on the recommendations of the reviewers and on my own reading, I think the paper could be published after rather moderate revision. Please address carefully all the specific comments of the reviewers.

    Sincerely,

    Fran\c{c}ois
  \end{quotation}

  
  \textbf{R:} We have taken the Reviewers' comments to heart and implemented all suggestions to the best of our ability. This includes adding a supplemental Mantel test and reporting all R$^2$ values as per Reviewer 1's request, refining our text as per Reviewer 2's suggestions, and addressing Reviewer 3's remaining questions in the discussion. We believe that these revisions have strengthened and clarified the manuscript and thank the Subject Editor and Reviewers for the thoughtful feedback.


\clearpage

{\large\textbf{Reply to Reviewer 1}} [[2/3 done]]

  \begin{quotation}

    In this manuscript the authors study by numerical simulation one of the numerous notion of “stability” of food webs. More precisely, they build food webs allometrically based on body size. Then they simulated the food webs and when a steady state is reached they removed, in turn, each of the constituting species. They, the authors measure the time of extinction of the other species. Finally, they study these extinctions as function of the role of the species in motifs and the role of the motif. In general it’s a well-written manuscript, easy to follow and that go quite deep into understanding the role of motifs on food-web dynamics. I think the paper deserved to be published, but I have few major comments that need to be addressed:

  \end{quotation}

  \textbf{R:} We appreciate the Reviewer's suggestions for alternative statistical tests and clearer presentation of our results, and have implemented these suggestions.
  % Kind of a dick, but easy to address. Fine.

  \smallskip

  \textbf{Major comments:}

    1. Confusion over $Z$-score normalisation [[Done (dummy)]]

      \begin{quotation}
        Line 171: the concept of “networks normalized” motif roles, defined as a $Z$-score is not well defined. Frankly speaking, I do not know how this was computed. Please be specific and provide the formula. Also, the notion of $Z$-score is strongly grounded in statistics, so please another terminology, as I guess here is it not related the Z-score as defined in statistics.
      \end{quotation}

      \smallskip

      \textbf{R:} We have added the formula for this normalisation as requested, and are confident that the Reviewer will recognise that this is indeed a $Z$-score as strongly grounded in statistics. As such, we have not changed our terminology. We hope that, with the addition of this equation, this normalisation is now clear.

      \begin{quotation}
        Finally, we also calculated `network-normalised' motif roles, defined as the $Z$-score ($Z_{imn}$) of a focal species $i$'s participation in motif $m$ compared to full set of species in network $n$:
        \begin{equation}
                Z_{imn} = \frac{x_{imn}-\mu_{mn}}{\sigma_{mn}} ,
        \end{equation}
        where $x_{imn}$ is the observed count of motif $m$ in the role of species $i$ in network $n$, $\mu_m$ is the mean count of motif $m$ in the roles of all species in network $n$, and $\sigma_{mn}$ is the standard deviation of the count of motif $m$ in network $n$.
      \end{quotation}

    \smallskip

    2. Suggestion to use a Mantel test instead of PERMANOVA [[In progress.]]

      \begin{quotation}
        PERMANOVAs: why to introduce this test if the assumptions are violated (line 208) ? In general I don’t think that PERMANOVA’s is the appropriated test here. The authors should use Mantel (with Spearman rank correlation) between the dissimilarly in species’s motif participation and the dissimilarity (distance) in time to extinction.
      \end{quotation}

      \smallskip

      \textbf{R:} We used the PERMANOVA test because it is well-established in the literature for relating motifs to other species or network properties. We present these results because we believe it is important to present negative results and data that doesn't conform to all statistical assumptions as well as the tests that come out perfectly, so long as it is clear where assumptions have been violated (and the Reviewer's comments suggest that we have been clear). 

      We have now performed the Mantel test as suggested. As the Mantel test does not allow stratified permutations (meaning that dissimilarities might be permuted between networks), we fit one test per network (6000 tests total). This makes the risk of obtaining significant results purely by chance even more severe than with the PERMANOVAs where we were able to fit one test per S-C combination (60 tests total). Although we again fit the correlated Bonferroni correction, these results still need to be taken with a large dose of salt. We therefore do not give them much space in the main text but present the full results in the SI for any reader who is curious.

    \smallskip

    3. Objection to presentation of $p$-values [[Done? Or does he want us to remove $p$-values from SI tables? Don't want to do that since someone would surely want them back. Add a note in the captions stating that number of simulations can change $p$-values maybe?]]

      \begin{quotation}
        Presentation of the results: on each figure legends the authors have to specify the R$^2$ (or partial  or marginal R$^2$) of fitted models. This is not done systematically. The authors also present the p-values of their linear mixed effects models. They are nonsense. Those p-values can be made arbitrarily small simply be increasing the number of simulations, i.e. they are simulations number dependent. The authors must present instead the R$^2$ (total, partial, or marginal). These values are simulations number independent, and more importantly, indicates the fraction of variability explained by the factors (tables S7 to S12, S14, S16, S18, S20, S22).
      \end{quotation}

      \smallskip

      \textbf{R:} As the Reviewer notes, the R$^2$ values were always available in SI tables and were, in the previous revision, also stated in the legends of Figs. 2 and 3. We also note that the R$^2$ value for Figure 4 was displayed in the figure itself. Within the main text, we present only $p$-values for the PERMANOVA tests which do not depend upon the number of simulations. For the linear mixed-effects models we presented only the R$^2$ in the main text and so do not understand the Reviewer's demand to remove $p$-values.


      Regarding the figure legends: We have now added the R$^2$ value to the legend of Figure 4 and removed it from the figure panel to avoid confusion. In addition, we have added references to the Appendix tables in the Figure legends. Also note that we have corrected the R$^2$ value in Figure 3 to match the marginal R$^2$ in the SI. We hope that this now provides sufficient documentation of our results.



\clearpage

{\large\textbf{Reply to Reviewer 2}} [[Minor done, need to add a definition(s) of stability(s) in introduction, point to extinction threshold as a caveat in results, earlier in discussion (not just alternative perspectives section), add list of model parameters (SI if bulky),]]

  \begin{quotation}
    I have read the manuscript by Wootton \& Cirtwill, and I have only minor comments to make. The problem statement is concise and definitely timely, the analyses are (plus or minus some suggestions I made) correct, and the conclusions are supported and broadly relevant.

    With the exception of these minor points, the paper is well referenced and enjoyable to read.
  \end{quotation}  

  \smallskip

  \textbf{R:} We thank the Reviewer for their comments and address each below. We hope that the manuscript is now clearer as a result

  \smallskip

  \textbf{Minor comments:}

    1. I would suggest removing the last sentence of the abstract. Nothing in the manuscript supports the idea of a causal relationship, and the sentence as written is too strongly worded.

    % Offending sentence:
    \begin{quotation}
      Previous research showed that certain motifs appear more frequently in stable webs; we show that this is at least in part because participating in these motifs provides protection from extinction.
    \end{quotation}

    \smallskip

    \textbf{R:} We have removed this sentence as suggested.

    \smallskip

    2. Missing ref. on line 72.

    \begin{quotation}
    (Donohue et al., 2013; Radchuk et al., 2019)
    \end{quotation}

    \smallskip

    \textbf{R:} In our compilation of the manuscript both references are shown. When submitting our revision we have taken careful note that this reference is shown.

    \smallskip

    3. In the introduction, I think a definition of stability (and related concepts) would help guide readers. This is a broad (and messy) literature, so a little disambiguation would go a long way.

    \smallskip

    \textbf{R:} % Slightly tricky to squeeze in but worth a shot.

    \smallskip

    4. One line 118, I think a more complete list of the (many) parameters of the model would help.

    % End of paragrpah summarizing network simulation
    % Requirements: 50-100 species (steps of 10), C 0.02-0.2 (steps of 0.02), no disconnected species or components, path lengths fully resolved.
    % Other model parameters baked into Julia? Is this what he wants?

    \smallskip

    \textbf{R:}

    \smallskip

    5. L. 121 - this is not quite a LV model, as explained in the original publication; instead, this is the Y\&I bioenergetic model adapted to multiple trophic levels. I would have also expected to see more discussion of rewiring, since the package now offers several models to do it.

    \begin{quotation}
    After generating the network structure, we simulated community dynamics using the function ``simulate'' from the Julia language package BioEnergeticFoodWebs (Delmas et al., 2019, 2017).
    \end{quotation}

    \smallskip

    \textbf{R:} Kate?

    \smallskip

    6. L. 129 - missing ref

    \smallskip

    \textbf{R:} We did not intend any citation in line 129. Possibly you mean the reference to Hairston \& Hairston in line 130? This appears in our compilation of the manuscript but we have taken care to double-check this reference when submitting our revision.%  (Hairston & Hairston, 1993)

    \smallskip

    7. L. 145 (and last section of the discussion) - this is probably something that should be implemented as a continuous callback. Furthermore, setting this empirical parameter should be accompanied with a small sensitivity analysis, as it can indeed have a strong effect on the results.

    \smallskip

    \begin{quotation}
    To ensure that species did not ‘recover’ from unrealistically low biomasses during the simulation, we considered a species extinct if it dropped below an arbitrary threshold biomass of 1×10$^{-5}$.
    \end{quotation}

    \textbf{R:} We regret that we are unable to implement a sensitivity analysis within the deadline for returning our revision. The computational time required for these dynamical models is a major drawback to the approach (and likely the reason why no-one has, to our knowledge, compared extinction thresholds to find a best-practice value). However, we are happy to make more frequent mention of this limitation and the possibility that it may have impacted our results. [[TO DO]]

\clearpage


{\large \textbf{Reply to Reviewer 3}} [[Minor addressed, Kate have a look at major \#1. Need to add reference to biomass effects in discussion]]

  \begin{quotation}
    Comments to the Author
    I think the authors have done an excellent job revising the manuscript and responding to reviewer suggestions. The methods are easier to follow and the main message is clear. I also definitely appreciate being able to look over the code, thank you for providing it.   

    While reading the revised manuscript I had two remaining questions for the authors that may be worth some mention in the discussion. 

  \end{quotation}

  \smallskip

  \textbf{R:} We thank the Reviewer for their previous and current comments. We have expanded the discussion along the lines suggested, while remaining conscious of the journal's word limit. We respond to the remaining questions below and appreciate the prompt to provide further nuance in our manuscript.

  \smallskip

  \textbf{Major comments} [[done?]]

    \begin{quotation}
      1. Are basal and non-basal species treated the same in this analysis? Because in these models basal species have an intrinsic growth rate, the probablity that they will go extinct during the simulation is likely not the same as a non-basal species. The results in the manuscript show that trophic level gives additional information about species' roles, and explains more variation in persistence than motif participation alone. Perhaps there is an interactive effect of trophic level and motif participation that could explain further variation.
    \end{quotation}

    \smallskip

    \textbf{R:} We agree that interactions between trophic level and motif participation (and degree) likely explain more variation in persistence than motif participation alone. Our goal was mainly to test the inference in many motif-analysis papers that motifs are related to stability, but moving forward combining motifs with other measures is definitely more promising than working with motifs alone. We have slightly expanded the discussion to state that we treated basal resources and consumers the same but that there is strong potential for interactions among different role measures. We hope that this will prompt readers to consider these interactions in future studies.


    \begin{quotation}
        Alternatively, future work exploring interactions between motifs and simple roles could improve our ability to predict extinction risk.
        For example, although we treated basal and non-basal species the same in this analysis, basal resources are not subject to all of the same threats as consumers.
        In particular, our simulation assigns basal resources a positive intrinsic growth rate to reflect the assumption that a basal resource will, having reached a particular habitat, have all of the resources (water, nitrogen, etc.) it needs. 
        In situations where basal resources this is unlikely (e.g., in highly degraded soils or drought-prone areas), basal resources may be at greater risk than in our simulations.
        As previous work shows that traits of both the disturbed and responding species affect extinction risk~\citep{Wootton2016} and that disturbance of basal resources can have particularly strong effects~\citep{Scherber2010}, adding the trophic levels of both disturbed and focal species to future analyses may improve our predictions.
        Likewise, interactions between motif participation and population biomasses could also affect how motifs relate to extinction risk.
        This could occur if species with high or low biomasses tend to appear in certain motifs more frequently or if participating in certain motifs assists species to be stable at different biomasses; future studies designed to separate these possibilities could strongly contribute to our mechanistic understanding of the relationships between network structure and persistence.
    \end{quotation}


    Since the Reviewer's question made us curious, we also had a slightly more detailed look at the extinction risk of basal resources in our study.
    As shown in Fig. 4, basal species tended to be less likely to go extinct than non-basal species. However, taking all networks together, basal and non-basal species had similarly variable mean times to extinction (see figure below), suggesting that other factors than trophic level also strongly influence each species' extinction risk. 

    \begin{minipage}{\textwidth}
        \centering
            \includegraphics[width=.5\textwidth]{figures/extorder_vs_TL.eps}
            \caption{Mean time to extinction was highly variable for all trophic levels.}
    \end{minipage}

     % On re-reading I'm not 100% sure how to test this. Maybe STL + motifs + STL:chain+STL:omni+STL...?
     % Or is a test even necessary? Perhaps we can just mention something in the discussion as they say in their intro

    \smallskip

    \begin{quotation}
      2. Do you think it would be important to consider initial biomass (at equilibrium prior to perturbation) when measuring time to extinction? The authors mention that persistence of each species depends on population biomass, and as a result assign initial biomass at random, but the equilibrium biomass may also be affecting time to extinction (and this may not be correlated with initial biomass after 1000 time steps). I imagine that species with relatively lower equilibrium biomass may go extinct faster than species with higher biomass given the same motif profile (assuming of course motif profile has no relationship to equilibrium biomass - but that is a question for another paper perhaps).   
    \end{quotation}

    \smallskip

    \textbf{R:} Whether motif profiles relate to equilibrium biomasses is indeed an interesting question, but as you say one for another paper. While the initial biomasses are random it is certainly possible that some motif profiles could be associated with species reaching high biomasses while other motif profiles are associated with species that only persist when rare. We have slightly expanded the discussion to suggest this as an interesting question for future research (see quotation above).


    Unfortunately we are not able to test whether equilibrium biomasses were related to time to extinction or to motif participation as we did not save the biomass data from our previous simulations and do not have time to repeat them before the revision is due. However, we share the Reviewer's intuition that species with lower biomasses are more likely to quickly go extinct. We have added a reference to possible biomass effects in the discussion and hope that this question will be resolved in future work.


    \smallskip


  \textbf{Minor comments} [[done]]

    3. L72: I think something messed up in the pdf conversion as there is "(??)" where I think a reference is supposed to go
    \begin{quotation}
    (Donohue et al., 2013; Radchuk et al., 2019)
    \end{quotation}

    \smallskip

    \textbf{R:} In our compilation of the manuscript both references are shown. When submitting our revision we have taken careful note that this reference is shown.

    \smallskip

    4. L77: I suggest changing "motifs in can" to "motifs it can"

    \smallskip

    \textbf{R:} Corrected.

    \smallskip

    5. L78: The phrasing "This is likely to lead species..." is a little awkward, perhaps switching to "Species with higher degrees likely participate..."

    \smallskip

    \textbf{R:} We have made the change as suggested. The sentence now reads ``Species with higher degrees can therefore likely participate in a wider variety of motifs, as well as participate more frequently in the four common, stable motifs."

    \smallskip

    6. L129: Another "(?)"

    \smallskip

    \textbf{R:} We did not intend any citation in line 129. Possibly you mean the reference to Hairston \& Hairston in line 130? This appears in our compilation of the manuscript but we have taken care to double-check this reference when submitting our revision.%  (Hairston & Hairston, 1993)

    \smallskip

    7. L359: Mentions "correlation of extinction orders" is this different from correlation of time to extinction?

    \smallskip

    \textbf{R:} No, this is the same concept. We have changed the line to read ``correlation of time to extinction" and thank the Reviewer for pointing out the inconsistency.

    \smallskip

    8. L397: I suggest changing "the omnivory can" by either removing "the" or adding "motif" (e.g., "the omnivory motif can")

    \smallskip

    \textbf{R:} We have added ``motif" and thank the Reviewer for catching this.

    \smallskip

    9. L405: I think it would be helpful if you could be more specific about what "strongly affected" means. In this case are you suggesting that more strongly affected = more likely to go extinct?

    % % Original:
    % \begin{quotation}
    % In small (3-5 species) model communities and soil food webs, weak interactions only promote stability if omnivory is present (Neutel et al., 2002; Emmerson \& Yearsley, 2004) and the loss of the omnivorous species dramatically reduces stability (Emmerson \& Yearsley, 2004). In our terms, if the top species in an omnivory motif is lost, the others will be strongly affected.
    % \end{quotation}

    \smallskip

    \textbf{R:} Yes, that is what we meant. We have revised the sentence to be more specific, as suggested. It now reads:
  
      \begin{quotation}
      In our terms, this suggests that when the top species in an omnivory motif is lost, the other species in the motif will tend to have substantially shorter times to extinction.
      \end{quotation}

    \smallskip

    10. L460: I think there may be a missing "or" between "predator prey"

    \smallskip

    \textbf{R:} Corrected.

    \smallskip

\clearpage

\end{letter}

\clearpage
    \bibliographystyle{ecollett}
 
    \bibliography{MyCollection} % Abbreviate journal titles.

\end{document}


