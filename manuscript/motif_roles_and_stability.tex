\documentclass[12pt]{article} 
\usepackage{amsmath} 
\usepackage[dvips]{graphicx}
\usepackage{multirow} 
\usepackage{geometry} 
\usepackage{pdflscape}
\usepackage[labelfont=bf]{caption} 
\usepackage{setspace}
\usepackage[running]{lineno} 
% \usepackage[numbers,sort]{natbib}
\usepackage[round]{natbib} 
\usepackage{array}
\usepackage[table]{xcolor}

\newcommand{\methods}{\textit{Materials \& Methods}}
\newcommand{\SI}{\textit{Appendix}~}

\topmargin -1.5cm % 0.0cm 
\oddsidemargin 0.0cm % 0.2cm 
\textwidth 6.5in
\textheight 9.0in % 21cm
\footskip 1.0cm % 1.0cm

\usepackage{authblk}

\title{Participation in stable motifs delays extinction in simulated food webs}

% - in intro/discussion, make it clear that this is adding a new layer of meaning to differences/changes in species roles rather than trying to put motifs as the best way to predict extinctions.

% 1st Theo Bio, then PeerJ. 
% Reviewers: Jon Borrelli, Benno Simmons, Daniel Stouffer, Tim Poisot, Eva Delmas
% To SI: PCA axes, extinction order correlations
% Main text: permanova shows that roles are important, participation to identify which motifs are most strongly assoc. with stability.
% Kate to work on coverletter, language edits
% Need abstract, ?graphical abstract?, author contributions,  
% Need 3-5 bullet point highlights, submitted as a separate file. 
% No obvious word limits?

\author{Alyssa R. Cirtwill$^{1\dagger}$, Kate Wootton $^{2}$} 
\date{\small$^1$Department of Ecology, Evolution, and Plant Sciences\\ 
Stockholm University\\
Stockholm, Sweden\\
\medskip
\small$^2$ Swedish Agricultural University\\
Uppsala, Sweden\\
\medskip
$^\dagger$ Corresponding author:\\
alyssa.cirtwill@gmail.com\\
 }

\renewcommand\Authands{ and }

\begin{document} 
\maketitle 
\raggedright
\setlength{\parindent}{15pt} 


\section{Keywords}

	species roles; disturbance; competition; three-species chain; omnivory


\section{Abstract}


\section{Introduction}

	The connections between food-web structure and stability have been of great interest to ecologists since at least the 1970's~\citep{May1972}. Initially, the focus was on identifying relationships between community size and stability (e.g.~\citealp{Gardner1970,May1972}). Given~\citet{May1972}'s finding that a large, randomly-connected network is unlikely to be stable, the ecological community quickly began to seek out non-random structural features that might confer stability or instability. These features include nestedness~\citep{Allesina2012,Sauve2014}, modularity~\citep{Sauve2014,Thebault2010} and distributions of link strengths~\citep{McCann1998,Gross2009,Rooney2012,Wootton2016}. Although important, these global-scale properties can mask important meso-scale differences in network structure~\citep{Simmons2019}. For instance, two networks with the same connectance and similar values of nestness and modularity may still have quite different arrangements of links. These differences, described by different frequencies of \emph{motifs} (unique patterns of $n$ interacting species) may also be related to network stability~\citep{Prill2005,Borrelli2015,Monteiro2016}. Indeed, empirical food webs tend to be composed of non-random sets of three-species motifs~\citep{Stouffer2007}. Empirical networks tend to contain more three-species chains and either more omivory motifs or more apparent and direct competition motifs than random networks~\citep{Stouffer2007}. A follow-up study showed that these four motifs make up, on average, about 95\% of all three-species motifs in food webs~\citep{Stouffer2010b}. 


	The high frequencies of three-species chain, omnivory, and apparent and direct competition motifs suggest that they may be beneficial to the network containing them. That is, it is possible that more stable motifs appear more frequently in empirical food webs because unstable motifs are more likely to disappear~\citep{Borrelli2015,Borrelli2015a}. The frequency of three-species chains and omnivory motifs do appear to be correlated with the overall persistence of  food webs, while the frequency of apparent and direct competition motifs does not~\citep{Stouffer2010}. Considering the stability of each motif in isolation, three-species chains and apparent and direct competition were very likely to be stable while omnivory is moderately likely to be stable~\citep{Borrelli2015a}. All other motifs are unlikely to be stable. It is possible, then, that three-species chains tend to be over-represented in empirical networks because they are very stable and unlikely to be removed through the loss of participating species or links~\citep{Borrelli2015}.  


	In summary, the frequencies with which different motifs appear in empirical networks appears to be related to the stability of those motifs in isolation~\citep{Stouffer2010b,Borrelli2015a} and the frequency of motifs within simulated networks is related to their probability of retaining all species following a perturbation~\citep{Stouffer2010b}. Thus, it is likely that the stability of these motifs and their contributions to network stability result in their over-representation in empirical networks~\citep{Borrelli2015}. Critically, this is not because the network as a whole is "adapted" for stability but because unstable motifs, and the species and interactions within them, are more likely to be "pruned" from empirical communities over time. We expect that stable structures will endure for longer than unstable ones, and so it is not surprising that stable structures should be over-represented in empirical communities~\citep{Borrelli2015}.


	Given the relationships between motif frequencies and whole-network stability, we may also expect that a species' role-- the frequency with which it participates in different positions within motifs --could affect its probability of extinction following a perturbation. As three-species chains, omnivory, and competition motifs are particularly stable in isolation~\citep{Borrelli2015a}, we might expect that species whose roles contain many such motifs might be less likely to go extinct than species whose roles are dominated by other motifs. Here, we investigate this question by simulating the removal of species from stable simulated networks. We test 1) whether species' roles overall are related to their time to extinction following a removal, 2) whether participation in particular motifs (especially the stable motifs described above) is correlated with time to extinction, and 3) whether these correlations are driven by a potential relationship between species' participation in various motifs and their numbers of interaction partners. Our overall aim is to establish whether changes in species' roles can, in future, be used to evaluate whether species are at increasing or decreasing risk of extinction.


	% Approx. 1000 words of introduction

	% Stability of biological networks has been related to the frequency of different motifs.
	% - Prill2005 transcription networks are more stable if they contain more "structurally stable" motifs (motifs whose structures mean a larger parameter space of signs and strengths is stable) motifs. Most stable motifs were apparent competition, direct competition, chain, omnivory (all fully structurally stable - will return to steady state if perturbed without oscillations), then quite a sharp drop before other motifs inc. two-way motifs and one-way loop. Two-way loop was least stable. Stable motifs are more abundant in biological networks. Three classes: I) acyclic graphs (no two-way links), fully stable; II) graphs with one two-way link; III) more complicated circuits (two-way link plus a loop, two two-way links, one-way loop, etc).
	% - Stouffer2007 showed chain, omnivory over-represented but both competitions under-represented compared to random in 10/16 empirical webs, competition over-represented and omnivory under in 6/16. Double-loop also over-represented... important to distinguish between abundance and statistical over-abundance of motifs. Motifs in empirical networks consistent with the niche model and not nested-hierarchy. 
	% - Stouffer2010 only relevant for pointer to Stouffer2007
	% - Stouffer2010b average 95\% of all motifs are chain, omnivory, apparent and direct competition. In isolation, species in tri-trophic chain most likely to all persist, then omnivory, then apparent competition, the direct competition. More species persist in food webs with many chains and omivory, persistence decreases with more apparent and direct competition.
	% - Stouffer2005 empirical connectance ranges 0.026-0.315 (25 to 155 trophic species)
	% - Stouffer2012 54 distinct roles across 32 webs, up to 22 roles per web (not related to size of web or taxonomic diversity). 46 roles for intermediate species, remainder are basal/int and int/top. Closely related species have similar roles, species with similar roles have similar benefits to their community (based on how much stability increases/decreases when a single motif is added). *Check SI for more details on benefit analysis.
	% - Kondoh2008 Intraguild predation modules can be stable if 1) prey is a superior competitor for the resource or 2) extra-module forces such as additional resources exist that benefit the prey more than the predator. In Caribbean food web, all but two species participate in at least 1 IGP module and three sharks are in \>1000. Intrinsically-stable modules were over-represented. Non-intrinsically stable modules were more likely to have external influences favouring prey. If externally-stablised IGP are stabilised by other IGP, may be vulnerable to perturbations. Expect more external stabilisation from internally-stable IGP. This is exactly what was found.
	% - Borrelli2015 Systems may be nonadaptively selected for stability by preferential removal of nonstable systems (e.g., culling interactions that lead to oscillations giving low abundances, )
	% - Rip2010 Modules that weaken interactions should stabilize the network, looking at "biparallel motif" as one such option - two chains linked at top and bottom. Tested empirically in microcosms, isolated. Generalist consumer de-synchronised its resources, stabilizing community. 
	% - Klaise2017 Disagree that the generalized niche model produces networks with similar motif profiles. Clustered food webs with similar profiles using a "hierarchical, agglomerative clustering algorithm based on the Pearson’s correlation coefficient (r)". Distance between webs is sqrt(2*(1-r)). Distances clustered using UPGMA (average linkage) algorithm. Trophic coherence (propensity of species to consume exclusively at the TL below themselves) tunes motif frequencies between two "families", only one of which is generated by the niche model. Main difference is whether omnivory is strongly over- or under-represented.
	% - Baiser2015 Local networks resemble regional ones if all trophic groups are sub-sampled with similar frequencies. Not particularly relevant.
	% - Monteiro2016  [Haydon1994, May1972, Sterner1997] show that modules are more stable with fewer interactions, small variance in interaction strengths, self-damping processes. Used Jacobian to identify stability of each 3- and 4-species motif, looked at frequencies in empirical networks. AC, DC, chain, and omnivory were all stable (omnivory could also be unstable). More stable motifs appeared more frequen tly. Same for 4-species motifs (4\% were stable). Omnivory ranging stable-unstable can explain contradictory results for omnivory in other studies.
	% - Gross2009 High variability in interaction strengths stabilizes small webs, destabilizes large ones. Stability also enhanced by generalist predators (high TL) and very vulnerable intermediate consumers. Used very large suite of simulated networks. No talk of motifs.
	% - Dambrot 2017 Having few loops seems to stabilize networks, but we don't know how they come to be un-loopy. Probably trophic coherence again. Not super relevant.
	% - Otto2007 Bioenergetic model to explore body-mass ratios and stability of tri-trophic chains. Only particular ratios are stable. Body mass also correlates with degree (+ with number of prey, - with number of predators)
	% - Bascompte2005 Ealier work (McCann1998) emphasizes stabilizing role of omnivory but nobody tested whether omnivory over-represented. Apparent competition and intraguild predation (4-sp, two chains with one middle sp. eating the other) more common than expected, omnivory more variable. 
	% - Rooney2012 Existence of fast and slow energy channels is stabilizing because it de-synchronises prey populations. Interactions in slow channels also tend to be weaker. Mostly about size and stability.
	% - Smith2015 Technical argument about how to standardize community matrices for calculating eigenvalues. Boring (sorry Gyuri)
	% - Allesina2008 Pred-prey networks are more likely to be stable than random ones. Varied off-diagonal C from 0.05 to 0.5 with steps of 0.025, size from 10 to 100 with steps of 5. 1000 webs of each size. Determined stability based on percentage of eigenvalues with negative real parts. Did not allow two-way links. Predator-prey networks are more stable than those where signs of interactions are random. Still stable if +/+ and -/- interactions present but less common than pred/prey. Weak interactions not necessary. Short cycles (self-regulation, predator-prey "cycles") may be key, will have greater influence bc. they are smaller. They believe they're supporting the "stable motifs lead to stable networks" lit.
	% - Borrelli2014 Food chains are generally 2-4 levels long. When weighted, usually 3 (Ulanowicz2013). Probably not because of inefficient transfer (productivity not strongly correlated with maxTL). Longer food chains take longer to stabilize following a perturbation. Looking at stability of mini-webs and larger random & niche model ones. Found that quasi sign-stability (a la Allesina2008) decreases with increasing chain length, dropping sharply after 3 for mini-webs to almost 0 for length 6.  
	% - Borrelli2015 This is the one about motifs and stability. Using Jacobian eigenvalue stability (return to equilibrium following small perturbation). Looked at sign stability of each motif in isolation, Z-scores in empirical webs. Chains, apparent competition, and direct competition were all more likely to be stable and over-represented. Omnivory was moderately likely to be stable, typically under-represented. Could indicate that unstable motifs are more likely to be "pruned", selecting for stability of the whole system. 
	% - German thesis Restricted to pairwise interactions and chains.
	% - Wootton2016a Pulse vs. press disturbance important. Used press disturbance because many anthropogenic effects thus, many affect one species disproportionately. Affected species may or may not be the one that goes extinct. Tested how web size and connectance affected resistance (size of disturbance required to cause extinction), how traits of disturbed species affected outcome. Found that smaller, less-connected networks were more resistant, focal extinctions more likely. 59\% of extinctions focal, 24\% of extinctions involve indirect interaction partners, 16\% of extinctions predators of focal species, 1\% of extinctions prey of focal species. Higher degree species were more llikely to cause extinction of predators, prey extinctions came from high S, high C webs with focal species with high degree. 


\section{Methods}

	\subsection*{Generating networks}

		We generated a suite of simulated networks based on the niche model~\citep{Williams2000}. The meso-scale structure of such networks closely mimics that of empirical food webs~\citep{Stouffer2007}. For greater realism, we specified a predator-prey body-mass ratio of 2.203 (derived from a ratio of log masses of 0.79) based on the global mean in~\citet{Brose2006}. To ensure that we captured a variety of realistic community sizes and structures, we generated networks ranging between 50 and 100 species (in steps of 10) with connectances between 0.02 and 0.2 (in steps of 0.02). We generated a total of 100 networks with each combination of parameters, for a total of 6000 networks. All networks were simulated using the function "nichemodel" within the Julia language package \emph{BioEnergeticFoodWebs}~\citep{bioenergeticfw,Delmas2017}. If a simulated network contained any disconnected species (species without predators or prey) or disconnected components (a group of species connected amongst themselves but not to the rest of the network), the network was rejected and a new network simulated. Finally, networks where the path lengths between each species and a basal resource could not be resolved were rejected and new networks simulated.


		Initial population biomasses were randomly assigned, with individual body masses assigned by the niche model. Following~\citet{Brose2006}'s estimates for vertebrates, we assumed a predator-prey bodymass ratio of 3.065 (the average of bodymass ratios for vertebrates across ecosystem and metabolism types). We excluded reported bodymass ratios for invertebrates as these could include parasites and parasitoids, which are generally smaller than their prey. 


		After assigning biomasses, we simulated community dynamics  in order to obtain an equilibrium community. If any species dropped below a threshold biomass of 1$\times10^{-5}$, we considered the species to have gone extinct. Community dynamics are based on Lotka-Volterra predator-prey models including density dependence and with type II functional responses for all species (please see~\citet{Delmas2017} for full details; all non-basal species were designated as vertebrates to ensure a good match between metabolic and predator-prey bodymass ratio values). Consumers were assumed to have no preferences such that the consumption rate $w_{ij}$ of $i$ eating $j$ is equal to $\frac{1}{n}$, where $n$ is the number of prey for predator $i$. If the network did not reach an equilibrium with all species persisting after 1000 time steps, a new set of initial population biomasses was applied and the simulation repeated. This process continued until an equilibrium state was achieved.


	\subsection*{Calculating motifs and species roles}

		We were interested in whether species' roles at equilibrium are related to their response to a perturbation, in this case a species removal. We defined species roles as their participation in unique positions within the set of three-species motifs, following~\citet{Stouffer2012,Cirtwill2015}. We expect that higher frequencies of positions in stable motifs (three-species chain, apparent and direct competition, and omnivory) will correlate with lower extinction risk while higher frequencies of positions in unstable motifs (those containing two- or three-species loops) will be related to higher extinction risk.	Note that cannibalistic links were ignored when calculating motif frequencies within a network and species' roles, although they were included when calculating connectance. As well as these 'raw' motif roles, we calculated normalized motif roles for each species by dividing the number of appearances in each motif position by the total number of times the species appears in any motif position. These 'normalized motifs' should control for differences in species' degrees (numbers of interaction partners).


		In addition to roles, we also calculated species' motif participation (i.e., the number of times a focal species appeared in any position within each motif). This provides a coarse-grained measure of how each species fits into the network, but is likely easier to relate to previous research than the more detailed motif roles. We normalized motif participation in two ways. Similar to motif roles, we first divided the raw participation counts by the total number of times the species participated in any motif ('degree normalisation'). In addition, we calculated a z-score of participation in each motif for the focal species compared to all species in its network ('network normalisation'). The degree normalisation allows us to test whether trends in stability with motif participation are due to differences in species' degrees, while the second normalisation allows us to test whether trends in stability with motif participation are related to how unusual each species is within its community context.


		% Apart from motif-based measures of species' positions in their communities, we calculated their degrees and short-weighted trophic levels. Degree is simply the number of interaction partners for each species. A species' short-weighted trophic level is the mean of its shortest trophic level an prey-averaged trophic level. The shortest trophic level is calculated based on the length of the shortest food chain between the focal species and any basal resource. Basal resources are assigned a trophic level of one and other species are assigned a shortest trophic level of one plus the trophic level of their prey. Prey-averaged trophic level is calculated based on the set of all shortest food chains between the focal species and any basal resources. The focal species' prey-averaged trophic level is the mean trophic level of its prey plus one. These alternative topological measures will be used to evaluate how strongly species' motif roles are related to their response to perturbation compared to other measures of position within a network. These measures were used to test how consistent motif roles and motif participation are for species which share the same degree or trophic level. [[Need to add this formally once I do it.]]


	\subsection*{Perturbing networks}

		After identifying species' roles in the equilibrium networks, we perturbed the networks by sequentially removing each species in each network. After each removal, community dynamics were simulated for 50 rounds of 10 time-steps (500 time-steps total). After each round, any species with a biomass below our threshold of 1$\times10^{-5}$ was considered to have gone extinct and its biomass was set to 0. We recorded the biomass of each species after each round, as well as the round in which any secondary extinctions occurred.


		For analyses of species' overall vulnerability, we used mean time to extinction across all removals as a response. Species which had longer mean times to extinction are understood to be more stable. To ensure that this is a robust measure, we calculated the Pearson correlation of times to extinction for each species across all extinctions in each network. We then tested whether the strength of these correlations varied with species richness and/or connectance by fitting a general linear model including fixed effects of species richness, connectance, and their interaction, as well as a random effect for network ID. We fit the model using the R~\citep{R} function 'lmer' from the package \emph{lmerTest}~\citep{lmerTest}.


	\subsection*{Testing effects of species' roles on time to extinction} [[Still need to update all of these, in dialogue with updating results. Some can likely be put into SI (PCA axes?)]]

		\subsubsection*{Roles and overall vulnerability}

			As a preliminary to testing whether species' roles are related to their order of extinction, we tested whether species' roles were consistent across networks. To do this, we conducted a principal component analysis (PCA) for the roles for all species in all networks for a given species richness and connectance. We then extracted the loadings of each position in each motif on the first three PCA axes. Using these loadings, we were able to visualise the motif positions which were most strongly associated with each PCA axis in each network.


			As well as visualising the associations between motif positions and major axes of variation in species roles across networks, we wanted to test whether these associations were consistent across networks with different species richnesses and connectances. We did this using a PERMANOVA, a multivariate analogue of classic ANOVA which, importantly, does not assume that the data follow any particular distribution. Instead, the data are permuted to generate a null distribution. We calculated the Euclidean distance between loadings of each position in each network and then tested whether these distances were more variable between or within positions.  We used 9999 unstratified permutations to construct the null distribution. 


			% [[Not a repeat: this is talking about a PERMANOVA of axis loadings~positions, below is extinction time~roles. ]]


			% After testing whether the importances of positions were consistent across networks, we then wished to test whether a species' mean time to extinction was related to the initial role of the removed species. To do this, we used a series of additional PERMANOVAs,  one for each combination of species richness and connectance. We calculated Bray-Curtis dissimilarities between the roles of removed and focal species as Bray-Curtis dissimilarity is not affected by double zeros (positions in which neither species appears)~\citep{Baker2015,Cirtwill2015}, and then tested whether these dissimilarities were related to species' mean time to extinction. We again used 9999 permutations to obtain the null distribution, but now stratified permutations within networks. That is, species roles could not be swapped between networks in order to preserve the distribution of roles within each network. As we performeed 60 unique PERMANOVAs (due to computational restrictions), we would expect approximately three positive results (with $\alpha$=0.05) simply due to multiple testing. We therefore evaluated the significance of each PERMANOVA using a Bonferroni correction (i.e., a threshold $p$-value of 0.00083 rather than 0.05). We performed all PERMANOVAs using the R~\citep{R} function 'adonis' from the package \emph{vegan}~\citep{vegan} using 9999 permutations to obtain the null  distribution.

			% [[I don't think it's worthwhile testing whether or not species went secondarily extinct... most seem to have.]]


			To support the PERMANOVAs relating time to extinction to motif roles, we fit a linear regression testing whether species' positions on the first three PCA axes were related to their mean time to extinction. For computational tractability, we first ordered species within each combination of network size and connectance by mean time to exinction, then divided species into subsets of 100 following~\citet{Simmons2019}. For each subset, we took the mean time to extinction and position on each PCA axis. We then fit a linear model relating mean time to extinction (per subset) to positions in each of the first three PCA axes, species richness, and connectance. We scaled and centered all predictors, and fit all models using the R~\citep{R} base function 'lm'.


			For an alternative perspective on how species' roles are related to their time to extinction, we considered species' participation in each motif (i.e., the number of times the species appears in any position in the motif). To account for the fact that some motifs, including the stable motifs in which we are most interested, tend to be more common than others~\citep{Stouffer2007}, we normalised motif participations across species. Rather than the simple count of the number of times a focal species appeared in a focal motif, we considered the Z-score of the focal species' participation compared to all other species in the network. These normalised participation scores indicate whether a species participates in a motif more (or less) frequently than others in the network. We then divided species into subsets of 100 as described above and fit a linear model relating mean time to exinction (per subset)  to mean normalised participation in each motif, species richness, and connectance.  We scaled and centered all predictors, and fit all models using the R~\citep{R} base function 'lm'.
			As it is likely that not all motifs explain substantial variation in species' roles, we simplified the model by sequentially removing non-significant terms beginning with the non-significant term with the smallest effect size. After each removal, we re-fit the model and checked for further non-significant terms.


		% \subsubsection*{Comparing the predictive ability of motifs and other traits}

		% 	We were also interested in whether the relationships between time to extinction and species motifs were similar to the relationships between time to extinction and other measures of topology. Here, we focused on degree (number of predators and prey) and shortest trophic level (defined as one plus the length of the shortest path to a basal resource; basal resources take the trophic level one). 


		% 	Could go about this in a couple of ways - predict based on degree, TL, and compare residuals with motif models, or do analyses like in Simmons2019 and see how much variability in roles there are for species with the same degree and TL?


\section{Results}

	\subsection*{Major axes of variation in species roles} [[updated]]
		[[I'm not sure if this is interesting. Maybe move to SI?]]

		The loadings of motif positions across the three major PCA axes were generally consistent across combinations of S and C ($F_{29,1770}$=30.7, $p$\textless0.001). 
		% stat_analysis/testing_role_consistency.R
		That is, a motif position which explains a lot of the variation in roles in one network is also likely to explain a lot of variation in roles in another, regardless of the size and connectance of the networks. 
		Within a combination of network size and connectance the first three PCA axes explained an average of 31.5\%, 14.6\%, and 12.1\% of variance, respectively. 
		[[Am I working with the average across size/connectance combos, or across all webs?]]
		The first PCA axis was most strongly and positively associated with positions 1, 11, 3, 9, and 5 and most strongly and negatively associateed with position 2 (all other positions had loadings with magnitude \textless0.1). 
		The second PCA axis was not strongly associated with any positions (all had loadings witth magnitude \textless0.1).
		The third PCA axis was most strongly and positively associated with positions 9, 12, and 5, and was not strongly and negatively associated with any other positions (all had loadings with magnitude \textless0.1).
		[[All of these are average loadings - is there a way to show the variation or is that too crowded?]]
		Note that positions 1-2 belong to the apparent competition motif, positions 3-5 to the three-species chain, positions 9-10 to the direct competition motif, and positions 11-13 to the omnivory motif. 


		% No binning
		Across all networks regardless of size and connectance, and adding network size and connectance as preictors, the first three PCA axes explained  51.1\%, 24.5\%, and 7.41\% of variance, respectively. %stat_testing/role_PCAs_forplots.R
		The first PCA axis was most strongly and positively associated with positions 1, 11, 3, 9, 14, and 5 and most strongly and negatively associated with position 2 (all other positions had loadings with magnitude \textless0.1). 
		Position 14 belongs to a motif including a loop (a direct competition motif with mutual predation between the two predators).
		All other positions associated with the first PCA axis are in stable motifs.
		The second PCA axis was most strongly an positively associated with positions 2, 13, 12, 4, 5, 9, 10, and 11 and was not strongly and negatively associated with any other positions (all other positions had loadings with magnitude \textless0.1). 
		The third PCA axis was most strongly and positively associated with posittions 9, 5, and 12 and most strongly and negatively associated with positions 4, 1, 13, 2, 14, and 10 (all other positions had loadings with magnitude \textless0.1). 
		Despite the contribution of position 14 to axes 1 and 3, most variation in species' roles appears to be explained by positions in the stable motifs whether we consider roles within a network class [[better replacement for size:connectance?]] or all roles together.
		% Did number of secondary extinctions vary with C, S? Seems important to quickly establish that to provide context.


	\subsection*{Roles and mean time to extinction} [[partly updated]]

		[[Introduced before above in methods. Rethink so that order is consistent.]]
		In general, time to extinction was highly correlated across removals (Fig.~\ref{extorder_corrs}). %figure_creation/extinction_order_correlations.py
		The mean Pearson correlation for times to extinction across all removals within a network was 0.903 (range: 0.512-0.973). % stat_analysis/mean_correlation_extorder_tests.R
		This means that, in general, the species which go extinct fastest when removing species $i$ also go extinct fastest when removing species $j$.
		The correlation was stronger in larger webs, particularly those with high connectance ($\beta_{S}$=1.33$\times10^-3$, $p$\textless0.001; $\beta_{C}$=-8.33$\times10^-2$, $p$\textless0.001; and $\beta_{S:C}$=2.16$\times10^-3$, $p$\textless0.001, respectively). 
		Mean time to extinction is, therefore, a good measure of a species' overall vulnerability.
		% [[Could look at which removals cause the biggest deviation from the mean. Are species which have particular roles/trophic levels causing particularly odd extinction timings?]]


		[[double-checked up to here]]
		[[This one is slow, waiting on it.]]
		% [[Roles are related to extinction order]]
		Across all combinations of species richness and connectance, species' roles were generally correlated with their mean time to extinction (Table~\ref{permtable}). Taken individually, each PERMANOVA was significant (all $p$\textless0.025). Moreover, after applying the correlated Bonferroni correction~\citep{Drezner2016}, all PERMANOVAs remained significant. While values of the pseudo-$F$ statistic increased steadily with increasing species richness, $p$-values were similar across values of species richness and connectance (Fig.~\ref{permfig}). There was no consistent interaction between species richness and connectance in either the pseudo-$F$ or $p$-values.


		% [[Add a figure with species on PCA axes, coloured by mean time to extinction? Going to be heckuva points.]]
		% Axis 1 is mostly apparent competition, so model that. Maybe regular and absolute values? 
		% -motif participation: in stable motifs (absolute number), all stable motifs together?
		% Axis 2 is top of chain (3) vs. apparent competition prey (2)
		% Axis 3 seems to be comparing apparent competition to DC, chain, omnivory.
		

		\subsubsection*{PCA axis positions and time to extinction} [[updated]]

			The first PCA axis was dominated by predator positions in the apparent competition and omnivory motifs, while the second PCA axis was dominated by the prey position in the apparent competition motif and the third PCA axis was dominated by the predator position in the direct competition motif and intermediate consumer position in the three-species chain (Fig.~\ref{PCA_plots}).
			[[Replace above description of PCA axes with this 1 sentence?]]	
			Species' positions on each of the three first PCA axes were related to their mean times to extinction. 
			% True if using values binned by extorder or all species. Using un-binned values because the bins don't appear necessary.
			Larger positive values of all three axes were assocciated with longer times to extinction ($\beta_{PC1}$=228, $p$\textless0.001; $\beta_{PC2}$=207, $p$\textless0.001; and $\beta_{PC3}$=7.29, $p$\textless0.001).
			Times to extinction were shorter in larger webs, more highly-connectetd webs, and especially in large and highly-connecctted webs ($\beta_{S}$=-25.4, $p$\textless0.001; $\beta{C}$=-0.224, $p$\textless0.001; and $\beta_{S:C}$=-0.608, $p$\textless0.001).
			The first PCA axis was dominated by predator positions in the apparent competition and omnivory motifs, while the second PCA axis was dominated by the prey position in the apparent competition motif and the third PCA axis was dominated by the predator position in the direct competition motif and intermediate consumer position in the three-species chain (Fig.~\ref{PCA_plots}).
			% Betas from regression using all roles/mean extinction times, no bins. Betas un-scaled.


		\subsubsection*{Motif participation and time to extinction} [[updated]]

			In the linear model including $Z$-scores of participation in all motifs as well as network size and connectance, all terms except for the $Z$-score of motif D7 were significant.
			After removing motif D7 and re-fitting the model, all terms were significant.
			Over-representation (higher $Z$-score) of the four stable motifs (S1, S2, S4, and S5) was associated witth longer times of extinctiton, as were over-representation of the one-way three-species loop (S3), D1, D3, and D6 motifs (Table~\ref{motif_lm}, Fig.~\ref{motif_coefs}. 
			Over-representation of motif S1, the three-species chain, was most strongly associated with increased time to extinction, followed by motif S4 (direct competition), motif D1 (direct competition with mutual competition among the two predators), and motif S5 (apparent competition).
			Of the four stable motifs, motif S2 (omnivory) was the least strongly associated with increased time to extinction.
			Under-representation (negative $Z$-score) of motifs D2, D4, D5, and D8 was associated with longer times to extinction.
			Greater network size and higher connectance were assocciated with shorter mean times to extinction ($\beta_{S}$=-9.81, $p$\textless0.001 and $\beta_{C}$=-0.157, $p$\textless0.001).

			[[It's a real pain having the positions out of order from the motifs, but I'm not sure I have the stomach to reorder the PCA motifs]]

			[[The PCA regression may not make sense on its own. More useful for showing that the stability effects aren't just because of the bigge range?]]

			\begin{table}
			\caption{Coefficients ($\beta$) and $p$-values from a linear model relating mean time to extinction to $Z$-scores of motif participation, species richness (S), and connectance (C). }
			\label{motif_lm}
			\begin{tabular}[h]{c | c | l l |}
			Motif & Description & $\beta$ & $p$-value \\
			\hline
			S & Species richness & -9.81 & \textless0.001 \\
			C & Connectance & -0.157 & \textless0.001 \\
			S1 & Chain & 1.23 & \textless0.001 \\
			S2 & Omnivory & 0.247 & \textless0.001 \\
			S3 & One-way loop & 0.0228 & 0.008 \\
			S4 & Direct competition & 0.878 & \textless0.001 \\
			S5 & Apparent competition & 0.672 & \textless0.001 \\
			D1 & S4 with mutual predation among predators & 0.846 & \textless0.001 \\
			D2 & S5 with mutual predation among prey & -0.388 & \textless0.001 \\
			D3 & S1 with mutual predation among top and intermediate species & 0.367 & \textless0.001 \\
			D4 & S1 with mutual predation among bottom and intermediate species & -0.981 & \textless0.001 \\
			D5 & S2 with mutual predation among top and bottom species & -0.350 & \textless0.001 \\
			D6 & Two-way three-species loop & 0.504 & \textless0.001 \\
			D7 & S2 with mutual predation between top and intermediate, intermediate and bottom species & \multicolumn{2}{c}{Not significant, removed} \\
			D8 & S1 with mutual predation between top and intermediate, intermediate and bottom species & -0.289 & \textless0.001 \\
			\hline
			\end{tabular}
			\end{table}



	Fig. 4 doesn't work super well, again no interactions with S and C.


	% \subsection*{Dissimilarity in roles and time to extinction}

	% 	[[not sure we'll include this]]

	% 	[[Role similarity, path length, and their interaction]]

	% 	[[Have yet to find a good way to summarise which p-values are significant, probably want to do a simplification exercise and remove interactions gradually...]]

	% 	In general, species in less-connected webs had longer times to extinction than species in more-connected webs. Time to extinction varied little as dissimilarity between the roles of the focal and removed species changed (Fig.~\ref{lmer_fig}A), and the slope of this trend could be either positive or negative as connectance and path length varied. In general, time to extinction decreased as the path length between the focal and removed species increased (Fig.~\ref{lmer_fig}B). We note that this contrasts with the results in [[Kate's thesis]], and may indicate that the model is over-fit [[or something]].
	% 	[[Maybe because Kate only looked at first secondary extinction? Should we zero in on just those?]]


	% 	[[Do two three-ways with S*C*path and S*C*dissim, compare? Do S and C as random effects for illustration purposes? ]]	

	% 	[[Does it matter how many motifs the focal and removed species share? Which ones?]]

\section{Discussion}

	- needs lots more ties to literature.


	We found that the positions explaining the most variation in species' roles were consistent across networks, that a species' time to extinction was consistent across species removals, and that species roles were significantly related to their time to extinction. More specifically, [[Give PCA, lmer results after re-run]]


	- roles relate to time to extinction
	- participation in more stable motifs leads to longer time to extinction
	- normalised roles ... 


	The positions explaining most variation in species' roles all came from one-way motifs, with apparent competition and three-species chains being particularly strongly associated with the first two PCA axes. This result is consistent with results in~\citet{Cirtwill2018EcolLett} which showed that the frequencies of positions in the three-species chain, apparent competition, and direct competition motifs explained the most variation in species' roles in highly-resolved marine food webs. These motifs are also those believed to be most stable~\citep{Stouffer2007,Borrelli2015a}. This is an encouraging result, as it suggests both that our simulated food webs show similar patterns of variation in species roles to empirical webs and that there is sufficient variation in the frequencies of stable motifs to be relevant for differences in species' times to extinction.


	We are also encouraged by the fact that species' times to extinction were tightly correlated across removals. This suggests that species' positions within a network influence their extinction risk, regardless of the identity of the removed species.  [[Discuss in relation to Kate's thesis, where removed species did matter]] The stronger correlation of extinction orders in larger, more-connected networks may be due to the greater number of pathways by which an extinction somewhere in the web can affect a focal species. As species richness increases, the length of these paths is also likely to increase, \emph{unless} connectance also increases~\citep{}. If these results hold true for empirical systems, then our findings suggest that it is more important \emph{that} extinctions are occurring than \emph{which} extinctions are occurring. Although this conflicts with the keystone species concept, which holds that some species are particularly important to the structure, stability, and/or functioning of their communities, it is consistent with other work identifying sets of characteristically vulnerable species (e.g., specialists, those with long generation times, top predators, etc.)~\citep{}.


	The results of the PERMANOVAs indicate that, across all combinations of species richness and connectance, species' roles are significantly associated with their times to extinction. Although the PERMANOVA does not indiciate which positions have the most explanatory power, we can use the PCA axes to infer that positions in the stable motifs are likely to be most important. In particular, the first PCA axis suggests that time to extinction increases with the frequency of positions at the top of the apparent competition, direct competition, and omnivory motifs, and with the top and middle positions in a three-species chain. This axis also suggests that mean time to extinction decreases with the frequency of the prey position in the apparent competition motif. All of these motifs have been identified as stable~\citep{Stouffer2007,Borrelli2015a}.

	
	The idea that appearing in more stable motifs promotes longer extinction times is also supported by the model selection on motif participation, which indicated that the species which participate in more three-species chain motifs than other species in their network tend to take longer to go extinct. Overall, then, our results indicate that species' roles within the meso-scale structure of their communities affect their responses to perturbations. If our 


	Our aim with this research is not to suggest that motif roles or participation are a superior predictor of extinction risk to degree or trophic levels. Instead, our goal is to expand the interpretation of motif roles. Previous research has shown that species' motif roles reflect their evolutionary history~\citep{} and traits such as foraging habitat and body size~\citep{Cirtwill2018EcolLett}. Our results demonstrate that species' roles are also relevant for their population dynamics after a perturbation. This gives researchers interested in predicting community responses to extinctions an additional tool to work with and expands the relevance of studies of motif roles.


\section{Acknowledgements}

	We thank Eva Delmas and Chris Rackauckas for their kind assistance with troubleshooting the simulation model. 


\section{References}

    \bibliographystyle{ecollett} 
    \bibliography{MyCollection} % Abbreviate journal titles.


\section{Tables}


	\begin{table}[h!]
		\caption{For each combination of species richness (S) and connectance (C), the mean extinction order of a focal species was related to its role. We tested this using a series of PERMANOVAs with 9999 permutations each. Here we show the mean correlation among extinction orders across all removed species ($R^2$) and all 100 simulated networks for each combination of S and C, as well as the pseudo-$F$ statistic and $p$-value for each PERMANOVA.}
		\label{permtable}
		\begin{tabular}{c c | c | c c ||c c | c | c c |}
			S	&	C	&	$R^2$	&	pseudo-$F$	&	$p$-value	&	S	&	C	$R^2$	&	pseudo-$F$	&	$p$-value\\ 
			\hline
			50&0.02&0.789&86.7&0.017\\ 
			50&0.04&0.813&66.4&0.013\\ 
			50&0.06&0.845&70.2&0.014\\ 
			50&0.08&0.843&77.4&0.015\\ 
			50&0.1&0.857&75.0&0.015\\ 
			50&0.12&0.868&104&0.020\\ 
			50&0.14&0.867&83.8&0.016\\ 
			50&0.16&0.872&89.7&0.018\\ 
			50&0.18&0.876&88.7&0.017\\ 
			50&0.2&0.880&103&0.020\\ 
			60&0.02&0.820&92.7&0.015\\ 
			60&0.04&0.846&86.0&0.014\\ 
			60&0.06&0.865&91.2&0.015\\ 
			60&0.08&0.872&99.7&0.016\\ 
			60&0.1&0.887&92.1&0.015\\ 
			60&0.12&0.883&96.9&0.016\\ 
			60&0.14&0.891&102&0.017\\ 
			60&0.16&0.890&106&0.017\\ 
			60&0.18&0.893&107&0.018\\ 
			60&0.2&0.899&137&0.022\\ 
			70&0.02&0.848&91.4&0.013\\ 
			70&0.04&0.875&108&0.015\\ 
			70&0.06&0.877&111&0.016\\ 
			70&0.08&0.898&112&0.016\\ 
			70&0.1&0.904&134&0.019\\ 
			70&0.12&0.907&124&0.017\\ 
			70&0.14&0.906&118&0.017\\ 
			70&0.16&0.909&122&0.017\\ 
			70&0.18&0.913&99.9&0.014\\ 
			70&0.2&0.917&122&0.017\\ 
			80&0.02&0.866&107&0.013\\ 
			80&0.04&0.898&114&0.014\\ 
			80&0.06&0.900&128&0.016\\ 
			80&0.08&0.908&147&0.018\\ 
			80&0.1&0.914&134&0.016\\ 
			80&0.12&0.915&140&0.017\\ 
			80&0.14&0.921&146&0.018\\ 
			80&0.16&0.923&125&0.015\\ 
			80&0.18&0.925&118&0.015\\ 
			80&0.2&0.926&123&0.015\\ 
			90&0.02&0.884&121&0.013\\ 
			90&0.04&0.906&142&0.016\\ 
			90&0.06&0.915&160&0.018\\ 
			90&0.08&0.923&151&0.017\\ 
			90&0.1&0.923&128&0.014\\ 
			90&0.12&0.927&128&0.014\\ 
			90&0.14&0.928&126&0.014\\ 
			90&0.16&0.931&138&0.015\\ 
			90&0.18&0.934&107&0.012\\ 
			90&0.2&0.936&127&0.014\\ 
			100&0.02&0.899&125&0.012\\ 
			100&0.04&0.917&191&0.019\\ 
			100&0.06&0.923&206&0.020\\ 
			100&0.08&0.932&176&0.017\\ 
			100&0.1&0.934&148&0.015\\ 
			100&0.12&0.934&156&0.015\\ 
			100&0.14&0.939&98.3&0.010\\ 
			100&0.16&0.938&144&0.014\\ 
			100&0.18&0.939&118&0.012\\ 
			100&0.2&0.942&105&0.010\\ 
			\hline
		\end{tabular}
		\end{table}

	% \begin{landscape}

\section{Figures}

	\begin{figure}[h!]
		\caption{\textbf{A-C)}Here we show the mean loadings of the 30 unique motif positions on the 3 first PCA axes. Positions which have loadings \textgreater0.0999 in at least one axis are labelled and coloured. Positions 1-2 make up the apparent competition motif, positions 3-5 make up the three-species chain, positions 9-10 make up the direct competition motif, and positions 11-13 make up the omnivory motif. 
		In \textbf{D)}, we show the 13 unique three-species motifs with positions numbered. Unique positions within each motif are indicated by different colours.}
		\label{PCA_plots}
		\includegraphics[height=.75\textheight]{figures/roles/roleplot_paper_points.eps}

	\end{figure}
			

	\begin{figure}[h!]
		\caption{In general, time to extinction for each species within a simulated network was highly correlated across removals. Circles indicate the mean correlation of time to extinction across removals for all species in all 100 simulated networks for a given combination of species richness and connectance. Lines indicate the predicted correlation based on the fixed effects of a linear model including species richness, connectance, and the interation between them, as well as a random effect of network. Symbol and line colours indicate connectance.}
		\label{extorder_corrs}
		\includegraphics[width=\textwidth]{figures/extinction_order/extorder_correlations_paper_full.eps}
		\end{figure}


	\begin{figure}[h!]
		\caption{Here we show (\textbf{A}) the psuedo-$F$ statistics and (\textbf{B}) the $p$-values for each PERMANOVA relating species' roles to their mean extinction order when all species in the web are separately removed. We fit one PERMANOVA per combination of species richness and connectance. $p$-values for each PERMANOVA are based on 9999 permutations, stratified by network. Symbols below the dotted line in \textbf{B} indicate a significant $p$-values.}
		\label{permfig}
		\includegraphics[height=.75\textheight]{figures/extinction_order/permanova_summary_paper_full.eps}
		\end{figure}

	% \begin{figure}[h!]
	% 	\caption{Here we show relationships between a species' position on the first three principal components axes and mean time to extinction. For interpretation of the axes, see Fig.~\ref{PCA_plots}.}
	% 	\label{PCA_lmers}
	% 	\includegraphics[height=0.75\textheight]{figures/extinction_order/PCA_position_lmer_summary_paper_full.eps}
	% 	\end{figure}

	\begin{figure}[h!]
		\caption{Here we show relationships between a species' participation in each motif and mean time to extinction, based on a linear model including $Z$-scores for the frequency of each motif as well species richness (S) and connectance (C). Shaded areas represent the four motifs known to be stable in isolation. Participation is defined as the number of times a species appears in the focal motif, regardless of position. Motif D7 was not significantly related to mean time to exttinction and was removed from the model. This is indicated by an 'x'. In order to display clearly differences among motifs, the coefficient for species richness (-9.81) is indicated by an arrow and not shown.}
		\label{motif_coefs}
		\includegraphics[width=\textwidth]{figures/extinction_order/motif_lmer_summary.eps}
		\end{figure}




	% \begin{figure}[h!]
	% 	\caption{Here we show how the predicted time to extinciton in a linear model including fixed effects of role dissimilarity, path length, connectance, and their interaction varies over (\textbf{A}) role dissimilarity, and (\textbf{B}) path length. Line colours indicate connectance. We fit separate models for each level of species richness (50-100 species, in steps of 10). [[Consider adding some indication of significance.]]}
	% 	\label{lmerfig}
	% 	\includegraphics[height=.75\textheight]{figures/extinction_order/dissimilarity_fits_summary_paper_full.eps}
	% 	\end{figure}



\end{document}



