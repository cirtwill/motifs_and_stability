\documentclass[12pt]{letter}

% \usepackage[britdate]{SU-letter}
\usepackage[britdate]{UH-letter}
\usepackage{times}
\usepackage{letterbib}
\usepackage{geometry}
\usepackage[round]{natbib}
\usepackage{graphicx}
\geometry{a4paper}
\usepackage[T1]{fontenc}
\usepackage[utf8]{inputenc}
\usepackage{authblk}
\usepackage[running]{lineno}
\usepackage{amsmath,amsfonts,amssymb}
% \usepackage[margin=10pt,font=small,labelfont=bf]{caption}

%\usepackage{natbib}
% \bibpunct[; ]{(}{)}{;}{a}{,}{;}

\newenvironment{refquote}{\bigskip \begin{it}}{\end{it}\smallskip}

\newenvironment{figure}{}

\position{Postdoctoral fellow}
\department{Department of Agricultural Sciences}
\location{University of Helsinki}
\cityzip{Helsinki, Finland}
\telephone{}
\fax{}
\email{alyssa.cirtwill@helsinki.fi}
\url{http://cirtwill.github.io}
\name{Dr. Alyssa R. Cirtwill}

\begin{document}

\begin{letter}{\bf Dr. Kathryn L. Cottingham \\
Department of Biological Sciences \\
Dartmouth College \\
Hanover, NH, USA}


\opening{Dear Prof. Cottingham:}

  My co-authors and I are happy to invite you to consider our manuscript
  \emph{Stable motifs delay species loss in simulated food webs} for publication in \textbf{Ecology}.


	Motifs --unique configurations of small numbers of interacting species-- are becoming more common as a tool for describing the meso-scale structure of ecological networks (i.e., a species 'neighborhood' within the network, including direct and close indirect interactions). 
	The set of motifs a species participates in has already been linked to species' traits, taxonomy, and location.
	More recent research has also suggested that some motifs, in isolation, dampen disturbances and facilitate persistence of the species within them while others do not.
	This suggests that species which participate in many of these `stable' motifs should be more likely to persist following a disturbance than species which participate in more of the `unstable' motifs.
	If so, motifs offer the possibility of estimating extinction risk within all types of ecological communities.


	Here we test this possibility using a large set of realistic simulated networks. In brief, we find that participating in more stable motifs is associated with longer persistence. Participating more often in the omnivory motif, however, was associated with shorter persistence. Motifs can therefore be added to the toolbox of researchers wishing to estimate species' risk of extinction. In particular, motifs offer a more detailed perspective on network structure than simpler measures such as degree (number of interaction partners) or trophic level and may be useful when comparing the risk of apparently similar taxa.


 	Our results also offer a potential explanation for the variable effects of omnivory on community stability. In our networks, participating more often in the omnivory motif was associated with greater extinction risk. However, species with high numbers of interaction partners (which tend to have lower extinction risk) appeared more often in the omnivory motif than species with few interaction partners. The apparent effect of omnivory may therefore depend on the balance between the effect of omnivory \emph{per se} and the effect of numbers of interaction partners.


 	Predicting which species are most likely to go extinct following a disturbance is one of the pressing ecological tasks of our time. 
 	As motifs provide a rich description of the ways species fit into their communities which can be applied to any ecological community, we are confident that establishing the relevance of motifs to extinction will be of interest to a broad section of ecologists.
 	Moreover, our results advance the basic science of network ecology by establishing the relationships between motif participation, degree, and trophic level. 
 	For both of these reasons, we believe that \textbf{Ecology} is the best place to publish this manuscript. 
 	We hope that you will agree and eagerly await your reply.


	The manuscript does not relate to any previous submission to an ESA journal. 

\closing{Best regards,}

	% Reviewers:

% Cover Letter. The cover letter should explain how the manuscript fits the scope of the journal, and more specifically how it advances the field, while having broad appeal. If the manuscript relates to any previous submission to an ESA journal, that must be explained as well. Longer submissions those between 30 and 50 manuscript pages) should be accompanied by a detailed justification for the length. There is a required text box for the cover letter. Uploading a cover letter as an attachment is optional.

\end{letter}
\end{document}


