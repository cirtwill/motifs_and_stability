\documentclass[12pt]{article} 
\usepackage{amsmath} 
\usepackage[dvips]{graphicx}
\usepackage{multirow} 
\usepackage{geometry} 
\usepackage{pdflscape}
\usepackage[labelfont=bf]{caption} 
\usepackage{setspace}
\usepackage[running]{lineno} 
% \usepackage[numbers,sort]{natbib}
\usepackage[round]{natbib} 
\usepackage{array}
\usepackage[table]{xcolor}

\newcommand{\methods}{\textit{Materials \& Methods}}
\newcommand{\SI}{\textit{Appendix}~}
\setcounter{table}{0}
\renewcommand{\thetable}{S\arabic{table}}%
\setcounter{figure}{0}
\renewcommand{\thefigure}{S\arabic{figure}}%

\topmargin -1.5cm % 0.0cm 
\oddsidemargin 0.0cm % 0.2cm 
\textwidth 6.5in
\textheight 9.0in % 21cm
\footskip 1.0cm % 1.0cm

\usepackage{authblk}

\title{Stable motifs delay species loss in simulated food webs: Appendices}

% - in intro/discussion, make it clear that this is adding a new layer of meaning to differences/changes in species motif participation rather than trying to put motifs as the best way to predict extinctions.

% 1st Theo Bio, then PeerJ. 
% Reviewers: Jon Borrelli, Benno Simmons, Daniel Stouffer, Tim Poisot, Eva Delmas
% To SI: PCA axes, extinction order correlations
% Main text: permanova shows that motif participation are important, participation to identify which motifs are most strongly assoc. with stability.
% Kate to work on coverletter, language edits
% Need abstract, ?graphical abstract?, author contributions,  
% Need 3-5 bullet point highlights, submitted as a separate file. 
% No obvious word limits?

% \author{Alyssa R. Cirtwill$^{1\dagger}$, Kate Wootton $^{2}$} 
% \date{\small$^1$Department of Agricultural Sciences\\ 
% University of Helsinki\\
% Helsinki, Finland\\
% \medskip
% \small$^2$ Swedish Agricultural University\\
% Uppsala, Sweden\\
% \medskip
% $^\dagger$ Corresponding author:\\
% alyssa.cirtwill@gmail.com\\
%  }

% \renewcommand\Authands{ and }
\date{}
\author{}

\begin{document} 
\maketitle 
\raggedright
\setlength{\parindent}{15pt} 

\section*{Table of Contents}

    \subsection*{S1: Details of network generation}

        Supplemental methods describing how the initial networks were generated and population dynamics simulated.

    \subsection*{S2: Testing consistency of times to extinction}
    
        Supplemental methods and results related to testing whether the identity of the removed species has a large effect on the time to extinction for non-removed species. Includes \textbf{Figure S1}.
    
    \subsection*{S3: Details of PERMANOVA methods and results}
        
        Supplemental methods and results for PERMANOVAs testing whether species' overall motif participation are related to their mean times to extinction. Includes \textbf{Figure S2, Tables S1-S2}.

    \subsection*{S4: Checking assumptions of PERMANOVAs}


    	Supplemental methods and results testing whether the assumption of equal dispersion of species motif participation across levels of persistence was met. Unequal dispersion can cause PERMANOVAs to return false positives, meaning that the results in Appendix \emph{S3} must be treated as tentative.


    \subsection*{S5: Pooling loop-containing motifs}

    	Supplemental methods detailing how the loop-containing motifs were pooled for each version of motif participation (counts of motifs, species normalisation, or network normalisation).


    \subsection*{S6: Relating motif participation to simple roles}

    	Supplemental ?methods? and results relating motif participation to a species' degree and trophic level.


    \subsection*{S7: Relating simple roles to persistence}

        Supplemental ?methods? and results relating degree and trophic level to persistence after a disturbance.


    \subsection*{S8: Motif labels}
        
        \textbf{Figure S3} illustrates the 13 unique three-species motifs and labels each one according to the naming scheme in~\citet{Stouffer2007}.

\clearpage
\begin{spacing}{2.0}
\linenumbers
\section*{S1: Details of network generation}


	We simulated a suite of food webs based on the probabilistic niche model, which assigns predator-prey links based on the body-mass ratios between individuals of different species~\citep{Williams2000,Delmas2017}. The meso-scale structure of niche-model networks closely mimics that of empirical food webs~\citep{Stouffer2007}. To ensure that we captured a variety of realistic community sizes and structures, we generated networks ranging between 50 and 100 species (in steps of 10) with connectance values between 0.02 and 0.2 (in steps of 0.02). The range of network sizes was chosen to reflect moderately well-sampled empirical webs while working within our computational limits, while the range of connectance values was chosen to cover that observed in most empirical food webs~\citep{Dunne2002e}. We generated a total of 100 networks with each combination of parameters, for a total of 6000 networks. All networks were generated using the function "nichemodel" within the Julia language package \emph{BioEnergeticFoodWebs}~\citep{bioenergeticfw,Delmas2017}. If a simulated network contained any disconnected species (species without predators or prey) or disconnected components (a group of species connected among themselves but not to the rest of the network), the network was rejected and a new network simulated. Finally, networks where the path lengths between each species and a basal resource could not be resolved (i.e., trophic levels were undefined) were rejected and new networks simulated.


	After generating the network structure, we simulated community dynamics using the function "simulate" from the Julia language package \emph{BioEnergeticFoodWebs}~\citep{bioenergeticfw,Delmas2017}. This function uses Lotka-Volterra predator-prey models including density dependence and type 2 functional responses for all species (please see~\citet{Delmas2017} for full details).
	All non-basal species were designated as vertebrates to ensure a good match between metabolic and predator-prey body-mass ratio values. Metabolic rates in the Lotka-Volterra model are based on each species' body mass (i.e. mass of a single individual). We assigned relative body masses based on each species' trophic level, which was, in turn, calculated based on the food-web structure provided by the niche model. After basal species were assigned a body mass of 1, we used a predator-prey body-mass ratio of 3.065 to calculate the relative body masses of higher trophic levels. We selected this ratio based on the estimate for vertebrates (averaged across ecosystem and metabolic types) in~\citet{Brose2006}. We excluded reported body-mass ratios for invertebrates as these could include parasites and parasitoids, which are generally smaller than their prey, and because interactions among vertebrates are better represented in the food-web literature than interactions involving invertebrates.
	
	
	The persistence of each species in our simulated networks also depends on its population biomass. 
	We randomly assigned initial population biomasses (i.e. cumulative biomass across all individuals of a species) for each species from a uniform distribution [0,1]. Note that population biomasses and individual body masses are not calculated on the same scale. We then simulated community dynamics for 1000 time steps to obtain an `equilibrium' community. To ensure that species did not `recover' from unrealistically low biomasses during the simulation, we considered a species extinct if it dropped below an arbitrary threshold biomass of 1$\times10^{-5}$. When simulating initial (i.e. pre-disturbance or equilibrium) dynamics, we rejected any network where one or more species dropped below this biomass threshold. Consumers were assumed to have no preferences such that the consumption rate $w_{ij}$ of predator $i$ eating prey $j$ is equal to $\frac{1}{n}$, where $n$ is the number of prey for predator $i$. If the network did not retain all species for 1000 time steps, a new set of initial population biomasses was applied and the simulation repeated.
	If a set of biomasses where all species persist still had not been reached after 100 sets of randomly-assigned initial biomasses, we discarded the network and simulated another to replace it.

\clearpage


\section*{S2: Testing consistency of times to extinction}

	\subsection*{Methods}

		We simulated population dynamics within each food web after removing each species separately. 
		In order to assess species' overall vulnerability to the loss of an arbitrary other species within the food web, we used mean time to extinction across all removals as a response. 
		To ensure that this is a robust measure, we calculated the Pearson correlation of times to extinction for each species across all extinctions in each network. 
		We then tested whether the strength of these correlations varied with species richness and/or connectance by fitting a general linear model including fixed effects of species richness, connectance, and their interaction, as well as a random effect for network ID. 
		We fit the model using the R~\citep{R} function `lmer' from the package \emph{lmerTest}~\citep{lmerTest}.


	\subsection*{Results}

		In general, time to extinction was highly correlated across removals (Fig.~\ref{extorder_corrs}). %figure_creation/extinction_order_correlations.py
		The mean Pearson correlation for times to extinction across all removals within a network was 0.903 (range: 0.512-0.973). % stat_analysis/mean_correlation_extorder_tests.R
		This means that, in general, the species which go extinct fastest after species $i$ is removed also go extinct fastest after species $j$ is removed.
		The correlation was stronger in larger webs, particularly those with high connectance ($\beta_{S}$=1.33$\times10^-3$, $p$\textless0.001; $\beta_{C}$=-8.08$\times10^-2$, $p$\textless0.001; and $\beta_{S:C}$=2.12$\times10^-3$, $p$\textless0.001, respectively). 
		Mean time to extinction is, therefore, a good measure of a species' overall vulnerability.
		% [[Could look at which removals cause the biggest deviation from the mean. Are species which have particular motif participation/trophic levels causing particularly odd extinction timings?]]


		\begin{figure}[h!]
			\caption{\textbf{A)} Time to extinction for each species within a simulated network was highly correlated across removals. Circles indicate the mean correlation of time to extinction across removals for all species in all 100 simulated networks for a given combination of species richness and connectance. Lines indicate the predicted correlation based on the fixed effects of a linear model including species richness, connectance, and the interaction between them, as well as a random effect of network. Symbol and line colors indicate connectance. \textbf{B)} Mean time to extinction across all species within a network was slightly longer in small and less-connected networks. As in \textbf{A)}, line colors indicate connectance. \textbf{C)} Mean time to extinction was more strongly associated with connectance than species richness, with more-connected networks having shorter mean times to extinction. Line colors indicate species richness.}
			\label{extorder_corrs}
			\includegraphics[width=.75\textwidth]{figures/extinction_order/extorder_correlations.eps}
			\end{figure}		

\clearpage


\section*{S3: Details of PERMANOVA methods and results}

	\subsection*{Methods}


		To test for a relationship between time to extinction and species' overall motif participation, we fit a series of PERMANOVAs~\citep{Anderson2001} relating Bray-Curtis dissimilarity~\citep{Baker2015,Cirtwill2015} in species' raw motif motif participation to differences in mean time to extinction (Table 1, \emph{Main Text}).
		Due to computational limits, we were unable to fit a single PERMANOVA for all networks.
		To avoid effects of network size and connectance on time to extinction, we therefore fit separate PERMANOVAs for each combination of network size and connectance (60 PERMANOVAs in total).
		We fit all PERMANOVAs using the R~\citep{R} function 'adonis' from the package \emph{vegan}~\citep{vegan} and calculated $p$-values using 9999 unstratified permutations.
		As conducting so many tests risks obtaining significant results by chance, we applied the correlated Bonferroni correction~\citep{Drezner2016} before evaluating significance.

	\subsection*{Results}


		Taken individually, each PERMANOVA was significant (all $p$\textless0.025). Moreover, after applying the correlated Bonferroni correction~\citep{Drezner2016}, all PERMANOVAs remained significant.
		All PERMANOVAs were significant, indicating that species' overall motif motif participation are related to their mean times to extinction.
		However, the significant betadisper results suggest that some of these PERMANOVA results may be false positives.
		After applying the correlated Bonferroni correction~\citep{Drezner2016}, both the ANOVAs testing for non-homogeneous variability of motif participation and regressions testing for relationships between role variability and time to extinction remained significant.


		\begin{figure}[h!]
			\caption{Here we show (\textbf{A}) the pseudo-$F$ statistics and (\textbf{B}) the $p$-values for each PERMANOVA relating species' motif participation to their mean extinction order when all species in the web are separately removed. We fit one PERMANOVA per combination of species richness and connectance. $p$-values for each PERMANOVA are based on 9999 permutations, stratified by network. Symbols below the dotted line in \textbf{B} indicate a significant $p$-values. }
			\label{permfig}
			\includegraphics[height=.5\textheight]{figures/extinction_order/permanova_summary_paper_full.eps}
			\end{figure}


		\begin{table}[h!]
			\caption{For each combination of species richness (S) and connectance (C), the mean extinction order of a focal species was related to its raw motif role. We tested this using a series of PERMANOVAs with 9999 permutations each. Here we show the mean of correlations of extinction orders for a focal species across all removed species ($R^2$) and all 100 simulated networks for each combination of S and C, as well as the pseudo-$F$ statistic and $p$-value for each PERMANOVA. All tests remained significant after applying the correlated Bonferroni correction~\citep{Drezner2016}.}
			\label{permtable}
			\begin{tabular}{c c | c | c c ||c c | c | c c |}
				S	&	C	&	$R^2$	&	pseudo-$F$	&	$p$-value	&	S	&	C &	$R^2$	&	pseudo-$F$	&	$p$-value\\ 
				\hline
				50	&	0.02	&	0.789	&	86.7	&	0.017	&	80	&	0.02	&	0.866	&	107	&	0.013	\\
				50	&	0.04	&	0.813	&	66.4	&	0.013	&	80	&	0.04	&	0.898	&	114	&	0.014	\\
				50	&	0.06	&	0.845	&	70.2	&	0.014	&	80	&	0.06	&	0.9	&	128	&	0.016	\\  
				50	&	0.08	&	0.843	&	77.4	&	0.015	&	80	&	0.08	&	0.908	&	147	&	0.018	\\
				50	&	0.10	&	0.857	&	75.0	&	0.015	&	80	&	0.10	&	0.914	&	134	&	0.016	\\
				50	&	0.12	&	0.868	&	104	&	0.020	&	80	&	0.12	&	0.915	&	140	&	0.017	\\
				50	&	0.14	&	0.867	&	83.8	&	0.016	&	80	&	0.14	&	0.921	&	146	&	0.018	\\
				50	&	0.16	&	0.872	&	89.7	&	0.018	&	80	&	0.16	&	0.923	&	125	&	0.015	\\
				50	&	0.18	&	0.876	&	88.7	&	0.017	&	80	&	0.18	&	0.925	&	118	&	0.015	\\
				50	&	0.20	&	0.88	&	103	&	0.020	&	80	&	0.20	&	0.926	&	123	&	0.015	\\
				60	&	0.02	&	0.82	&	92.7	&	0.015	&	90	&	0.02	&	0.884	&	121	&	0.013	\\
				60	&	0.04	&	0.846	&	86.0	&	0.014	&	90	&	0.04	&	0.906	&	142	&	0.016	\\
				60	&	0.06	&	0.865	&	91.2	&	0.015	&	90	&	0.06	&	0.915	&	160	&	0.018	\\
				60	&	0.08	&	0.872	&	99.7	&	0.016	&	90	&	0.08	&	0.923	&	151	&	0.017	\\
				60	&	0.10	&	0.887	&	92.1	&	0.015	&	90	&	0.10	&	0.923	&	128	&	0.014	\\
				60	&	0.12	&	0.883	&	96.9	&	0.016	&	90	&	0.12	&	0.927	&	128	&	0.014	\\
				60	&	0.14	&	0.891	&	102	&	0.017	&	90	&	0.14	&	0.928	&	126	&	0.014	\\
				60	&	0.16	&	0.89	&	106	&	0.017	&	90	&	0.16	&	0.931	&	138	&	0.015	\\
				60	&	0.18	&	0.893	&	107	&	0.018	&	90	&	0.18	&	0.934	&	107	&	0.012	\\
				60	&	0.20	&	0.899	&	137	&	0.022	&	90	&	0.20	&	0.936	&	127	&	0.014	\\
				70	&	0.02	&	0.848	&	91.4	&	0.013	&	100	&	0.02	&	0.899	&	125	&	0.012	\\
				70	&	0.04	&	0.875	&	108	&	0.015	&	100	&	0.04	&	0.917	&	191	&	0.019	\\
				70	&	0.06	&	0.877	&	111	&	0.016	&	100	&	0.06	&	0.923	&	206	&	0.020	\\
				70	&	0.08	&	0.898	&	112	&	0.016	&	100	&	0.08	&	0.932	&	176	&	0.017	\\
				70	&	0.10	&	0.904	&	134	&	0.019	&	100	&	0.10	&	0.934	&	148	&	0.015	\\
				70	&	0.12	&	0.907	&	124	&	0.017	&	100	&	0.12	&	0.934	&	156	&	0.015	\\
				70	&	0.14	&	0.906	&	118	&	0.017	&	100	&	0.14	&	0.939	&	98.3	&	0.010	\\
				70	&	0.16	&	0.909	&	122	&	0.017	&	100	&	0.16	&	0.938	&	144	&	0.014	\\
				70	&	0.18	&	0.913	&	99.9	&	0.014	&	100	&	0.18	&	0.939	&	118	&	0.012	\\
				70	&	0.20	&	0.917	&	122	&	0.017	&	100	&	0.20	&	0.942	&	105	&	0.010	\\
				\hline
			\end{tabular}
			\end{table}

\clearpage


\section*{S4: Checking assumptions of PERMANOVAs}

	\subsection*{Methods}

        
        In addition to participation in particular motifs affecting a species' vulnerability to extinction, it may also be that those species most vulnerable to extinction (i.e. short time to extinction) have more variable motif participation than those which are least vulnerable, or vice versa. 
        This could occur if, for example, species with a particular level of persistence tend to participate very frequently in a single motif but not always the same motif (e.g., apparent competition or direct competition but not both). 
        Such motif participation vectors would have a larger Bray-Curtis dissimilarity than motif participation vectors of species which participate more evenly in all motifs.


        If there is unequal variance in motif participation across different times to extinction, this could cause false positive results in PERMANOVA tests.
        To account for this possibility, we first calculated the dispersion of motif participation for each value of mean time to extinction (treated categorically for this analysis), relative to the centroid for all species with the same mean time to extinction, using the function `betadisper' from the R~\citep{R} package \emph{vegan}~\citep{vegan}.
        We then tested whether some levels of mean time to extinction are associated with more widely-dispersed motif participation using an ANOVA test, fit using the function `anova' within the package \emph{vegan}~\citep{vegan}.
        Finally, to test whether role dispersion generally increases or decreases with increasing mean times to extinction, we fit a linear model relating role dispersion to mean time to extinction using the R~\citep{R} base function `lm'.


	\subsection*{Results}

        For all three ways of defining motif motif participation (counts, species-normalization, and network-normalization), motif motif participation were not homogeneously variable across deciles of mean time to extinction (Tables~\ref{betadisp_count},~\ref{betadisp_species},~\ref{betadisp_network}). 
        For many combinations of network size and connectance, variability increased significantly at higher deciles.
        This means that there are a wider variety of motif motif participation associated with long times to extinction than short times to extinction.
        However, the difference in variability between deciles did not obviously increase between more distant deciles (e.g., deciles 2 and 10 vs. deciles 2 and 3; Fig.~\ref{betadisper_Tukey}).
        Moreover, the absolute differences in variability were small.
        Therefore, despite the violation of some assumptions of the PERMANOVA tests we performed, we nevertheless view the results as highly suggestive.
        

        \begin{figure}[h!]
            \centering
            \includegraphics[height=.75\textheight]{figures/Tukey_differences.eps}
            \caption{Variability of motif motif participation was not homogeneous between groups, but the difference in variability between deciles of mean time to extinction did not obviously increase with the difference between deciles and the absolute differences in variability between groups were low. We show the difference in variability between all pairs of deciles in all PERMANOVAs (60 PERMANOVAs per motif version. Significant differences are indicated in red and juddered left; non-significant differences are indicated in blue and juddered right.}
            \label{fig:my_label}
        \end{figure}


		\begin{table}[h!]
			\caption{For each combination of species richness (S) and connectance (C), some levels of mean time to extinction were associated with more variable motif participation than others. This may cause false positives in the PERMANOVAs reported in table~\ref{permtable}. In all cases, the variability of species' motif participation increased significantly with increasingly long mean times to extinction. All tests remained significant after applying the correlated Bonferroni correction~\citep{Drezner2016}.}
			\label{disptable}
			\footnotesize
			\begin{tabular}{c c | c c| c c ||c c | c c | c c |}
				&		&	\multicolumn{2}{c|}{ANOVA}	&\multicolumn{2}{c||}{Regression} 			& & & 	\multicolumn{2}{c|}{ANOVA}		 	&	 \multicolumn{2}{c|}{Regression} 			\\
	            S	&	C	&	$F$	&	$p$-value	&	$\beta$	&	$p$-value	&	S	&	C	&	$F$	&	$p$-value	&	$\beta$	&	$p$-value	\\
				\hline
	        50	&	0.02	&	3.38	&	\textless0.001	&	3.36$\times10^{-4}$	&	\textless0.001	&	80	&	0.02	&	3.72	&	\textless0.001	&	1.74$\times10^{-4}$	&	\textless0.001	\\
	        50	&	0.04	&	3.90	&	\textless0.001	&	2.60$\times10^{-4}$	&	\textless0.001	&	80	&	0.04	&	3.87	&	\textless0.001	&	1.34$\times10^{-4}$	&	\textless0.001	\\
	        50	&	0.06	&	3.75	&	\textless0.001	&	2.12$\times10^{-4}$	&	\textless0.001	&	80	&	0.06	&	3.86	&	\textless0.001	&	1.15$\times10^{-4}$	&	\textless0.001	\\
	        50	&	0.08	&	3.59	&	\textless0.001	&	1.81$\times10^{-4}$	&	\textless0.001	&	80	&	0.08	&	3.83	&	\textless0.001	&	1.04$\times10^{-4}$	&	\textless0.001	\\
	        50	&	0.10	&	3.56	&	\textless0.001	&	1.76$\times10^{-4}$	&	\textless0.001	&	80	&	0.10	&	3.59	&	\textless0.001	&	8.39$\times10^{-5}$	&	\textless0.001	\\
	        50	&	0.12	&	3.76	&	\textless0.001	&	1.60$\times10^{-4}$	&	\textless0.001	&	80	&	0.12	&	3.49	&	\textless0.001	&	8.00$\times10^{-5}$	&	\textless0.001	\\
	        50	&	0.14	&	3.39	&	\textless0.001	&	1.52$\times10^{-4}$	&	\textless0.001	&	80	&	0.14	&	3.27	&	\textless0.001	&	7.80$\times10^{-5}$	&	\textless0.001	\\
	        50	&	0.16	&	3.34	&	\textless0.001	&	1.34$\times10^{-4}$	&	\textless0.001	&	80	&	0.16	&	3.02	&	\textless0.001	&	6.34$\times10^{-5}$	&	\textless0.001	\\
	        50	&	0.18	&	3.23	&	\textless0.001	&	1.28$\times10^{-4}$	&	\textless0.001	&	80	&	0.18	&	2.72	&	\textless0.001	&	5.46$\times10^{-5}$	&	\textless0.001	\\
	        50	&	0.20	&	3.12	&	\textless0.001	&	1.24$\times10^{-4}$	&	\textless0.001	&	80	&	0.20	&	2.55	&	\textless0.001	&	4.11$\times10^{-5}$	&	\textless0.001	\\
	        60	&	0.02	&	3.58	&	\textless0.001	&	2.78$\times10^{-4}$	&	\textless0.001	&	90	&	0.02	&	3.56	&	\textless0.001	&	1.52$\times10^{-4}$	&	\textless0.001	\\
	        60	&	0.04	&	3.88	&	\textless0.001	&	2.05$\times10^{-4}$	&	\textless0.001	&	90	&	0.04	&	4.00	&	\textless0.001	&	1.13$\times10^{-4}$	&	\textless0.001	\\
	        60	&	0.06	&	3.72	&	\textless0.001	&	1.74$\times10^{-4}$	&	\textless0.001	&	90	&	0.06	&	3.64	&	\textless0.001	&	9.23$\times10^{-5}$	&	\textless0.001	\\
	        60	&	0.08	&	3.47	&	\textless0.001	&	1.50$\times10^{-4}$	&	\textless0.001	&	90	&	0.08	&	3.44	&	\textless0.001	&	7.63$\times10^{-5}$	&	\textless0.001	\\
	        60	&	0.10	&	3.58	&	\textless0.001	&	1.23$\times10^{-4}$	&	\textless0.001	&	90	&	0.10	&	3.24	&	\textless0.001	&	7.03$\times10^{-5}$	&	\textless0.001	\\
	        60	&	0.12	&	3.45	&	\textless0.001	&	1.29$\times10^{-4}$	&	\textless0.001	&	90	&	0.12	&	3.02	&	\textless0.001	&	5.14$\times10^{-5}$	&	\textless0.001	\\
	        60	&	0.14	&	3.45	&	\textless0.001	&	1.12$\times10^{-4}$	&	\textless0.001	&	90	&	0.14	&	2.85	&	\textless0.001	&	5.56$\times10^{-5}$	&	\textless0.001	\\
	        60	&	0.16	&	3.30	&	\textless0.001	&	1.11$\times10^{-4}$	&	\textless0.001	&	90	&	0.16	&	2.92	&	\textless0.001	&	5.26$\times10^{-5}$	&	\textless0.001	\\
	        60	&	0.18	&	3.07	&	\textless0.001	&	1.03$\times10^{-4}$	&	\textless0.001	&	90	&	0.18	&	2.58	&	\textless0.001	&	4.11$\times10^{-5}$	&	\textless0.001	\\
	        60	&	0.20	&	3.15	&	\textless0.001	&	9.84$\times10^{-5}$	&	\textless0.001	&	90	&	0.20	&	2.59	&	\textless0.001	&	4.17$\times10^{-5}$	&	\textless0.001	\\
	        70	&	0.02	&	3.45	&	\textless0.001	&	2.16$\times10^{-4}$	&	\textless0.001	&	100	&	0.02	&	3.99	&	\textless0.001	&	1.31$\times10^{-4}$	&	\textless0.001	\\
	        70	&	0.04	&	3.76	&	\textless0.001	&	1.64$\times10^{-4}$	&	\textless0.001	&	100	&	0.04	&	3.95	&	\textless0.001	&	9.81$\times10^{-5}$	&	\textless0.001	\\
	        70	&	0.06	&	3.83	&	\textless0.001	&	1.46$\times10^{-4}$	&	\textless0.001	&	100	&	0.06	&	3.65	&	\textless0.001	&	8.23$\times10^{-5}$	&	\textless0.001	\\
	        70	&	0.08	&	3.64	&	\textless0.001	&	1.11$\times10^{-4}$	&	\textless0.001	&	100	&	0.08	&	3.31	&	\textless0.001	&	6.52$\times10^{-5}$	&	\textless0.001	\\
	        70	&	0.10	&	3.39	&	\textless0.001	&	9.72$\times10^{-5}$	&	\textless0.001	&	100	&	0.10	&	3.11	&	\textless0.001	&	5.62$\times10^{-5}$	&	\textless0.001	\\
	        70	&	0.12	&	3.42	&	\textless0.001	&	9.64$\times10^{-5}$	&	\textless0.001	&	100	&	0.12	&	3.05	&	\textless0.001	&	5.07$\times10^{-5}$	&	\textless0.001	\\
	        70	&	0.14	&	3.09	&	\textless0.001	&	8.62$\times10^{-5}$	&	\textless0.001	&	100	&	0.14	&	2.54	&	\textless0.001	&	3.22$\times10^{-5}$	&	\textless0.001	\\
	        70	&	0.16	&	3.05	&	\textless0.001	&	8.31$\times10^{-5}$	&	\textless0.001	&	100	&	0.16	&	2.56	&	\textless0.001	&	3.44$\times10^{-5}$	&	\textless0.001	\\
	        70	&	0.18	&	2.85	&	\textless0.001	&	6.25$\times10^{-5}$	&	\textless0.001	&	100	&	0.18	&	2.58	&	\textless0.001	&	3.63$\times10^{-5}$	&	\textless0.001	\\
	        70	&	0.20	&	2.81	&	\textless0.001	&	6.77$\times10^{-5}$	&	\textless0.001	&	100	&	0.20	&	2.32	&	\textless0.001	&	2.17$\times10^{-5}$	&	\textless0.001	\\

	    \end{tabular}
	    \end{table}

\clearpage


\section*{S5: Pooling loop-containing motifs [[Do these make sense?]]}
	
	The motifs containing loops (two-way interactions or three-species loops) are rare in both empirical systems~\citep{StoufferXXXX} and our simulated networks (means 8.99$\times10^{-4}$\%-2.06\% of the total count of motifs for a species) and are all unstable when modelled in isolation~\citep{BorrelliXXXX}.
	To more clearly identify any effects of participating in rare but unstable motifs, we pool these loop-containing motifs into an 'other' group.
	This reduces our set of `motifs' to five and greatly simplifies interpretation of our analyses.
	The manner in which the loop-containing motifs differed slightly depending on the version of motifs considered (detailed below).


	\subsection*{Participation based on motif counts}

		For the non-normalised motif participation vectors, which were lists of the number of times a species appeared in each motif, we simply summed the counts of all loop-containing motifs in each species' participation vector.
		This produces a vector of length 5 where entries are the counts of the number of times the focal species appears in the apparent competition motif, direct competition motif, omnivory motif, and three-species chain motif separately and, finally, all loop-containing motifs taken together.


	\subsection*{Participation based on species-normalisation}

		The species-normalisation version of motif participation removed any potential effects of species participating in different total counts of motifs.
		This is accomplished by dividing each count by the total number of times a species appears in any motif; the resulting vector of frequencies sums to 1 for all species.
		To pool the loop-containing motifs, we summed the frequencies for all such motifs.
		The resulting vector is of length 5 and has a similar structure to the count vector above but sums to 1 in all cases.


	\subsection*{Participation based on network normalisation}

		The network-normalisation version of motif participation considers whether a species participates in many or few of each motif relative to the rest of its community.
		Participation in each motif is included as a $Z$-score of the count for the focal species relative to the mean and standard deviation of counts for that motif within the network.
		To pool the loop-containing motifs, we first summed the counts of these motifs as when pooling the non-normalised motifs.
		We then calculated $Z$-scores of participation in the pooled `other' motifs.
		This tells us whether a focal species participates in unusually few or many loop-containing motifs overall.


\clearpage


\section*{S6: Relating motif participation to simple roles}

	\subsection*{Methods}

        For ease of interpretation, we once again grouped the loop-containing motifs into an `unstable' group, leading to 6 LMs of the form:

        \begin{equation}
            \phi_{in} \approx \alpha_{i} + \delta_{i} + o_{i} + \chi_{i} + \omega_{i} + S_{n}:C_{n} + N_n,
            \label{eq:degTL_motifs}
        \end{equation}

        where $\phi_{in}$ is the degree or trophic level of species $i$ belonging to network $n$, $\alpha_{i}$, $\delta_{i}$, $o_{i}$, and $\chi_{i}$ are the species' participation in the apparent competition, direct competition, omnivory, and three-species chain motifs (respectively), $\omega_{i}$ is the species' participation in the unstable motif group, $S_{n}:C_{n}$ is a random effect of the size and connectance of network $n$, and $N_n$ is a random effect of belonging to network $n$.

	\subsection*{Results}

		\textbf{Degree}

			Degree increased significantly with the count of each motif (Table~\ref{tab:motifs_vs_deg}).
			Taken together, motif counts explained \textgreater90\% of the variation in degree. 
			As a species has more interactions, it naturally participates in more of each type of motif.
			This increase was especially strong for the `other' and three-species chain motifs (Fig.~\ref{fig:motif_counts_vs_deg}).


			\begin{table}[h!]
			\caption{A species' degree was strongly correlated with its motif participation. This association was especially strong for un-normalised (count-based) motif participation.}
			\label{tab:motifs_vs_deg}
			\begin{tabular}{l | c c | c c | c c}
			& \multicolumn{2}{|c|}{Counts} & \multicolumn{2}{c}{Frequencies} & \multicolumn{2}{|c}{$Z$-scores} \\
			Motif & $\beta$ & $p$-value & $\beta$ & $p$-value & $\beta$ & $p$-value \\
			\hline
			Intercept &  3.84  & \textless0.001 & 83.3 & \textless0.001 & 22.3 & \textless0.001\\
			\hline
			Apparent competition       &  0.0194  & \textless0.001 & -83.4 & \textless0.001 & 1.88 & \textless0.001 \\
			Direct competition       &  0.0249  & \textless0.001 & -61.5 & \textless0.001 & 1.80 & \textless0.001\\
			Omnivory       &  0.0274  & \textless0.001 & 27.6 & \textless0.001 & 5.69 & \textless0.001\\
			Three-species chain       &  0.0388  & \textless0.001 & -90.4 & \textless0.001 & 1.64 & \textless0.001\\
			Other    &  0.0470  & \textless0.001 & \multicolumn{2}{c|}{NA} & 4.61 & \textless0.001 \\
			\hline
			$R^2$ & \multicolumn{2}{|c|}{$R^2_m$=0.926, $R^2_c$=0.977} & 
			\multicolumn{2}{c}{$R^2_m$=0.629, $R^2_c$=0.740} & 
			\multicolumn{2}{|c}{$R^2_m$=0.468, $R^2_c$=0.896} \\
			\end{tabular}
			\end{table}


			Degree increased significantly with an increasing frequency of the omnivory motif, but decreased significantly with increasing frequencies of the apparent competition, direct competition, and three-species chain motifs (Table~\ref{tab:motifs_vs_deg}).
			Species-normalised motif participation explained less variation in degree ($\approx$ 60\%) than count-based motif participation, although the two role definitons are still strongly linked.
			[[One-sentence ref to figure (Fig.~\ref{fig:motif_freq_vs_deg}).]]


			As with counts, degree increased significantly with increasing $Z$-scores of all motifs (Table~\ref{tab:motifs_vs_deg}).
			Network-normalised motif participation explained the least variation in degree ($\approx$ 50\%), although the two role definitons are still strongly linked.
			The increase in degree was strongest with increases in the omnivory motif and `other' motifs (Fig.~\ref{fig:motif_Z_vs_deg})


		\textbf{STL (shortest trophic level)}


			[[Is it worth unpacking these much, since the R2 are so extremely low?]]


			\begin{table}[h!]
			\caption{}
			\label{tab:motifs_vs_STL}
			\begin{tabular}{l | c c | c c | c c}
			& \multicolumn{2}{|c|}{Counts} & \multicolumn{2}{c}{Frequencies} & \multicolumn{2}{|c}{$Z$-scores} \\
			Motif & $\beta$ & $p$-value & $\beta$ & $p$-value & $\beta$ & $p$-value \\
			\hline
			Intercept & 2.26 & \textless0.001 & 2.90 & \textless0.001 & 2.13 & \textless0.001 \\
			\hline
			Apparent competition & 6.22$\times10^{-5}$ & \textless0.001 & -0.612 & \textless0.001 & 2.47$\times10^{-2}$ & \textless0.001 \\
			Direct competition   & -1.56$\times10^{-4}$ & \textless0.001 & -0.460 & \textless0.001 & 1.88$\times10^{-3}$ & \textless0.001 \\
			Omnivory       & -5.76$\times10^{-4}$ & \textless0.001 & -1.86 & \textless0.001 & -9.98$\times10^{-2}$ & \textless0.001 \\
			Three-species chain  & -4.18$\times10^{-4}$ & \textless0.001 & -0.715 & \textless0.001 & -3.14$\times10^{-2}$ & \textless0.001 \\
			Other    & 1.28$\times10^{-4}$ & \textless0.001 & \multicolumn{2}{c|}{NA} & 2.54$\times10^{-2}$ & \textless0.001 \\
			\hline
			$R^2$ & \multicolumn{2}{|c|}{$R^2_m$=0.044, $R^2_c$=0.193} & 
			\multicolumn{2}{c}{$R^2_m$=0.030, $R^2_c$=0.157} & 
			\multicolumn{2}{|c}{$R^2_m$=0.021, $R^2_c$=0.110} \\
			\end{tabular}
			\end{table}



		Tables, figures, etc.


\clearpage



\section*{S7: Relating simple roles to persistence}

	\subsection*{Methods}

            To complete the context for our results, we fit a regression of persistence against degree, trophic level, and their interaction, as well as a random effect of global network structure:
            
            \begin{equation}
                \tau_{in} \approx \Delta_{i} + \Upsilon_{i} + \Delta_{i}\Upsilon_{i} + S_{n}:C_{n} + N_n ,
                \label{eq:persistence_degTL}
            \end{equation}
            
            where $\tau_{in}$ is the mean persistence time for species $i$, belonging to network $n$, $\Delta_i$ is the degree of species $i$, $\Upsilon_i$ is the shortest trophic level of species $i$, $S_{n}:C_{n}$ is a random effect of the size and connectance of network $n$, and $N_n$ is a random effect of belonging to network $n$.
            These regressions are intended only to confirm whether our simulations show the expected increase in persistence with degree and decrease in persistence with trophic level.

	\subsection*{Results}

		As expected, persistence was significantly related to degree and trophic level (Table~\ref{tab:per_degTL}, Fig.~\ref{fig:persistence_degTL}).


		\begin{table}[h!]
			\caption{Effect sizes ($\beta$) and $p$-values for fixed effects in a LMM of persistence (mean time to extinction) against degree, shortest trophic level, and their interaction. The LMM also included a random effect of global network structure (the combination of network size and connectance). We also give the amount of variance explained by fixed effects alone ($R^2_m$) and all fixed and random effects ($R^2_c$).}
			\label{tab:per_degTL}
			\begin{tabular}{l | c c |}
			Predictor & $\beta$ & $p$-value \\
			\hline
			Intercept & 49.1 & \textless0.001 \\
			Degree & 0.466 & \textless0.001 \\
			STL & -7.72 & \textless0.001 \\
			Interaction & -0.174 & \textless0.001 \\
			\hline
			$R^2$ & \multicolumn{2}{|c}{$R^2_m$=0.211, $R^2_c$=0.232} \\
			\end{tabular}
			\end{table}


	    \begin{figure}[h!]
	        \centering
	        \includegraphics[width=\textwidth]{figures/roles/persistence_vs_degTL.eps}
	        \caption{Mean persistence time increased with increasing degree for species at low trophic levels (STL 1-2) but decreased with increasing degree for species at high trophic levels (STL \textgreater2). The figure is based on the fixed effects in a regression of persistence against degree, trophic level, their interaction, and a random effect of global network structure.}
	        \label{fig:persistence_degTL}
	    \end{figure}


\clearpage


\section*{S8: Relating motif participation to persistence}
	
	Position of appendix depends on whether there are methods for S6 or S7.


	\subsection*{Non-normalised (count) motif participation}

		In a LMM of persistence against all motifs (including a random effect of global network structure), motif participation explained a small amount of variation ($R^{2}_m$=0.058; $R^2_c$=0.220).
		In this model, greater participation in all motifs except for the omnivory motif was associated with \emph{greater} persistence (Table~\ref{tab:persistence_motifs}).
		The effect of participation in the omnivory motif was, however, an order of magnitude smaller than the effects of participation in the apparent competition and direct competition and two orders of magnitude smaller than the effect of participation in the three-species chain motif.
		The effect of participation in the loop-containing motifs was also small, similar to the omnivory motif.


		Participation in the omnivory motif was strongly and negatively correlated with participation in the apparent competition and chain motifs, and weakly and negatively correlated with participation in the direct competition and `other' motifs (Table~\ref{tab:count_correlations}).
		Participation in the `other' motifs was strongly and positively correlated with participation in the direct competition motif.
		Participation in the apparent competition motif was moderately and positively correlated with participation in the three-species chain motif.


		[[Add a nice simple figure.]]


		\begin{table}[h!]
		\caption{Here we give the estimate ($\beta$) and $p$-value for a LMM of persistence (mean time to extinction) against participation in each motif and a random effect of global network structure. Motif participation was defined as the non-normalised count; the number of times the focal species appeared in each motif. We also give the variance explained by each regression. R$^{2}_{m}$ gives the variance explained by fixed effects only, while R$^{2}_{c}$ gives the variance explained by fixed and random effects. We calculated variance explained using the R~\citep{R} function `r.squaredGLMM' from the package \emph{MuMIn}~\citep{MuMIn}.}
		\label{tab:persistence_motifs}
		\begin{tabular}{l | c c}
		 Motif & $\beta$ & $p$-value \\  
		 \hline
		 Intercept & 29.7 & \textless0.001 \\
		 \hline
		 Apparent competition & 7.77$\times10^{-3}$ & \textless0.001 \\
		 Direct competition & 8.05$\times10^{-3}$ & \textless0.001 \\
		 Omnivory &  -3.91$\times10^{-4}$ & 0.133 \\
		 Three-species chain &  1.73$\times10^{-2}$ & \textless0.001 \\
		 Other &  5.76$\times10^{-4}$ & 0.039 \\
		 \hline
		 Variance explained & \multicolumn{2}{c}{$R^{2}_m$=0.058; $R^2_c$=0.220} \\

		 \hline
		 \end{tabular}
		 \end{table}


		[[Convert all 3 correlation tables into a 3-panel heat map?]]


		\begin{table}[h!]
		\caption{Correlation amongst counts of each motif in a species' non-normalised participation vector. The 8 loop-containing motifs have been grouped into a single `other' category.}
		\label{tab:count_correlations}
		\begin{tabular}{l | c c c c}
			& AC & DC & Omni & Chain \\
		\hline
		DC    & -0.085 &        &        &        \\   
		Omni  & -0.592 & -0.256 &        &        \\
		Chain &  0.311 & -0.196 & -0.417 &        \\
		Other & -0.143 &  0.397 & -0.293 & -0.117 \\
		\hline
		\end{tabular}
		\end{table}

% # Alpha order of names is S5, S4, S2, S1
	\subsection*{Species-normalised (frequency) motif participation}


		Higher frequencies of participation in the apparent competition, omnivory, and three-species chain motifs were associated with longer persistence, while participation in the direct competition motif was not significantly associated with persistence (Table~\ref{tab:persistence_freq}).
		Species-normalised motif participation explained much less variance in persistence than un-normalised participation.
		Frequencies of the stable motifs were strongly and positively correlated (Table~\ref{tab:freq_correlations}).
		Frequency of participation in the `other' motifs (not included in the LMM due to model rank deficiency) was moderately and positively correlated with participation in the omnivory motif and negatively correlated with participation in the remaining stable motifs.


		Add a simple figure.


		\begin{table}[h!]
		\caption{Here we give the estimate ($\beta$) and $p$-value for a LMM of persistence (mean time to extinction) against participation in each motif and a random effect of global network structure. Motif participation was defined as the frequency with which the species appeared in each motif, normalised by the total number of motifs the species appeared in. Because the frequencies for each species must sum to 1, leading to a rank-deficient model, the `other' motifs were not included in the LMM. We also give the variance explained by each regression. R$^{2}_{m}$ gives the variance explained by fixed effects only, while R$^{2}_{c}$ gives the variance explained by fixed and random effects. We calculated variance explained using the R~\citep{R} function `r.squaredGLMM' from the package \emph{MuMIn}~\citep{MuMIn}.}
		\label{tab:persistence_freq}
		\begin{tabular}{l | c c}
		 Motif & $\beta$ & $p$-value \\  
		 \hline
		 Intercept & 30.7 & \textless0.001 \\
		 \hline
		 Apparent competition & 2.20 & \textless0.001 \\
		 Direct competition & 0.255 & 0.528 \\
		 Omnivory & 13.9 & \textless0.001 \\
		 Three-species chain & 3.68 & \textless0.001 \\
		 \hline
		 Variance explained & \multicolumn{2}{c}{$R^{2}_m$=0.005; $R^{2}_c$=0.106} \\
		 \hline
		 \end{tabular}
		 \end{table}


		\begin{table}[h!]
		\caption{Correlation amongst counts of each motif in a species' species-normalised participation vector. The 8 loop-containing motifs have been grouped into a single `other' category. The `other' motifs were not included in the LMM; correlations between the four stable motifs and the pooled `other' motifs were calculated separately.}
		\label{tab:freq_correlations}
		\begin{tabular}{l | c c c c}
			& AC & DC & Omni & Chain \\
		\hline
		DC    &  0.753 &       & & \\         
		Omni  &  0.737 &  0.588 &   & \\   
		Chain &  0.872 &  0.692 & 0.640 & \\
		Other & -0.299 & -0.428 & 0.251 & -0.100 \\
		\hline
		\end{tabular}
		\end{table}


	\subsection*{Network-normalised ($Z$-score) motif participation}

		Having a higher $Z$-score in the omnivory motif was associated with lower persistence, while having a higher $Z$-score in any other motif was associated with greater persistence (longer mean time to extinction; Table~\ref{tab:persistence_Z}).
		These trends explained less variation in persistence than count-based motif participation, but more than frequency-based motif participation.
		$Z$-scores for participation in the omnivory motif were strongly and negatively correlated with $Z$-scores for participation in the apparent competition and three-species chain motifs, and weakly and negatively correlated with $Z$-scores for participation in the direct competition and `other' motifs (Table~\ref{tab:Z_correlations}).
		[[More commentary needed?]]


		\begin{table}[h!]
		\caption{Here we give the estimate ($\beta$) and $p$-value for a LMM of persistence (mean time to extinction) against participation in each motif and a random effect of global network structure.  Motif participation was defined as the $Z$-score of participation in each motif, relative to the species' community. We also give the variance explained by each regression. R$^{2}_{m}$ gives the variance explained by fixed effects only, while R$^{2}_{c}$ gives the variance explained by fixed and random effects. We calculated variance explained using the R~\citep{R} function `r.squaredGLMM' from the package \emph{MuMIn}~\citep{MuMIn}.}
		\label{tab:persistence_Z}
		\begin{tabular}{l | c c}
		 Motif & $\beta$ & $p$-value \\  
		 \hline
		 Intercept & 34.7 & \textless0.001 \\
		 \hline
		 Apparent competition & 1.52 & \textless0.001 \\
		 Direct competition &  0.462  & \textless0.001 \\
		 Omnivory & -0.359  & \textless0.001 \\
		 Three-species chain & 1.50 & \textless0.001 \\
		 Other &  0.197  & \textless0.001 \\
		 \hline
		 Variance explained & \multicolumn{2}{c}{$R^{2}_m$=0.022; $R^{2}_c$=0.108} \\
		 \hline
		 \end{tabular}
		 \end{table}


		\begin{table}[h!]
		\caption{Correlation amongst counts of each motif in a species' network-normalised participation vector. The 8 loop-containing motifs have been grouped into a single `other' category. The `other' motifs were not included in the LMM; correlations between the four stable motifs and the pooled `other' motifs were calculated separately.}
		\label{tab:Z_correlations}
		\begin{tabular}{l | c c c c}
			& AC & DC & Omni & Chain \\
		\hline
		DC     & -0.165 &        &        &    \\
		Omni   & -0.598 & -0.176 &        &     \\
		Chain  &  0.216 & -0.268 & -0.423 &      \\
		Other  & -0.169 &  0.399 & -0.195 & -0.199 \\
		\hline
		\end{tabular}
		\end{table}

\clearpage













\section*{Old PLSR material}

	The PERMANOVAs indicate whether motif participation as a whole are related to mean time to extinction but do not reveal which motifs have the strongest effect on time to extinction.
	To answer this question, we used a set of partial least squares (PLS) regressions to identify combinations of motifs which, together, explain substantial variation in time to extinction. 
	Similar to a principal components analysis (PCA), PLS projects the observed variables (i.e., participation in different motifs) into a new space and identifies latent variables made up of linear combinations of the observed variables~\citep{Mevik2004,pls} which, together, explain substantial variation in mean time to extinction.
	
	
	First, we used mean time to extinction as the response and raw motif motif participation as well as network size, connectance, the interaction between size and connectance, in-degree (number of prey), and shortest trophic level (STL)~\citep{Hairston1993} as predictors (Table~\ref{overview_table}).
	We include additional measures of network structure and species motif participation as these may affect the motif motif participation available to each focal species.
	Next, to test whether any relationship between raw motif motif participation and time to extinction might be due differences in the motif participation of species appearing in different total numbers of motifs, we fit a second PLS regression using degree-normalized motif participation instead of the raw motif participation, with all other variables identical to the first regression.
	Third, to understand whether it is the absolute frequency of motifs or the relative frequency compared to other species in the network that is related to time to extinction, we fit a final PLS regression using network-normalized motif participation as a predictor and all other variables identical to the first regression.
	
	
	We fit all regressions using the R~\citep{R} function 'plsr' from the package \emph{pls}~\citep{pls}.
	To prevent differences in range and intercept values from influencing the fit of the PLS model, we centered and scaled all variables.
	After initial fitting, we cross-validated each regression using 10 randomly-selected segments of the data and re-fitting the regression and then calculated the mean squared error of prediction (MSEP) for each model.
	MSEP is a measure of the error obtained when re-fitting a PLS or PCA model on test data, and is commonly used to select the optimum number of components~\citep{Mevik2004}.
	In order to balance obtaining a low MSEP with identifying a parsimonious model, we defined the optimum model as that with the fewest components that nevertheless had an MSEP within one standard deviation of the lowest MSEP obtained for any model.
	Model selection was performed using the R~\citep{R} function 'selectNcomp' from the package \emph{plsr}~\citep{pls} using the method `onesigma'.
	After selecting the optimum number of components for each model, we re-fit the PLS regression including only the selected components. 
	We then summed the coefficients of each predictor across axes to obtain an overall measure of the effect of each predictor on mean time to extinction.

\clearpage


\section*{S9: Motif labels}

	\begin{figure}[h!]
		\caption{There are 13 unique three-species motifs which can appear in food webs. Motifs S1, S2, S4, and S5 have been identified as more stable than other motifs when modeled in isolation and as potentially increasing the stability of empirical food webs. Labels are as in~\citet{Stouffer2007}.}
		\label{motifs}
		\includegraphics[width=.8\textwidth]{figures/motifs.eps}
		\end{figure}

\end{spacing}
\clearpage

    \bibliographystyle{ecollett} 
    \bibliography{MyCollection} % Abbreviate journal titles.


\end{document}



