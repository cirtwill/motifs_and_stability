\documentclass[12pt]{article} 
\usepackage{amsmath} 
\usepackage[dvips]{graphicx}
\usepackage{multirow} 
\usepackage{geometry} 
\usepackage{pdflscape}
\usepackage[labelfont=bf]{caption} 
\usepackage{setspace}
\usepackage[running]{lineno} 
% \usepackage[numbers,sort]{natbib}
\usepackage[round]{natbib} 
\usepackage{array}
\usepackage[table]{xcolor}

\newcommand{\methods}{\textit{Materials \& Methods}}
\newcommand{\SI}{\textit{Appendix}~}
\setcounter{table}{0}
\renewcommand{\thetable}{S\arabic{table}}%
\setcounter{figure}{0}
\renewcommand{\thefigure}{S\arabic{figure}}%

\topmargin -1.5cm % 0.0cm 
\oddsidemargin 0.0cm % 0.2cm 
\textwidth 6.5in
\textheight 9.0in % 21cm
\footskip 1.0cm % 1.0cm

\usepackage{authblk}

\title{Stable motifs delay species loss in simulated food webs: Appendices}

% - in intro/discussion, make it clear that this is adding a new layer of meaning to differences/changes in species roles rather than trying to put motifs as the best way to predict extinctions.

% 1st Theo Bio, then PeerJ. 
% Reviewers: Jon Borrelli, Benno Simmons, Daniel Stouffer, Tim Poisot, Eva Delmas
% To SI: PCA axes, extinction order correlations
% Main text: permanova shows that roles are important, participation to identify which motifs are most strongly assoc. with stability.
% Kate to work on coverletter, language edits
% Need abstract, ?graphical abstract?, author contributions,  
% Need 3-5 bullet point highlights, submitted as a separate file. 
% No obvious word limits?

% \author{Alyssa R. Cirtwill$^{1\dagger}$, Kate Wootton $^{2}$} 
% \date{\small$^1$Department of Agricultural Sciences\\ 
% University of Helsinki\\
% Helsinki, Finland\\
% \medskip
% \small$^2$ Swedish Agricultural University\\
% Uppsala, Sweden\\
% \medskip
% $^\dagger$ Corresponding author:\\
% alyssa.cirtwill@gmail.com\\
%  }

% \renewcommand\Authands{ and }
\date{}
\author{}

\begin{document} 
\maketitle 
\raggedright
\setlength{\parindent}{15pt} 

\section*{Table of Contents}

    \subsection*{S1: Details of network generation}

        Supplemental methods describing how the initial networks were generated and population dynamics simulated.

    \subsection*{S2: Testing consistence of times to extinction}
    
        Supplemental methods and results related to testing whether the identity of the removed species has a large effect on the time to extinction for non-removed species. Includes \textbf{Figure S1}.
    
    \subsection*{S3: Details of PERMANOVA methods and results}
        
        Supplemental methods and results for PERMANOVAs testing whether species' overall roles are related to their mean times to extinction. Includes \textbf{Figure S2, Tables S1-S2}.

    \subsection*{S4: Details of PLSR methods}

        Supplemental methods for partial least squares regressions. 
    
    \subsection*{S5: Relating motifs roles to other measures}
        
        Gives the slopes of relationships between motif roles and degree or shortest trophic level (STL) in \textbf{Tables S3-S4}.
        
    \subsection*{S6: Motif labels}
        
        \textbf{Figure S3} illustrates the 13 unique three-species motifs and labels each one according to the naming scheme in~\citet{Stouffer2007}.

\clearpage
\begin{spacing}{2.0}
\linenumbers
\section*{S1: Details of network generation}


	We simulated a suite of food webs based on the probabilistic niche model, which assigns predator-prey links based on the body-mass ratios between individuals of different species~\citep{Williams2000,Delmas2017}. The meso-scale structure of niche-model networks closely mimics that of empirical food webs~\citep{Stouffer2007}. To ensure that we captured a variety of realistic community sizes and structures, we generated networks ranging between 50 and 100 species (in steps of 10) with connectance values between 0.02 and 0.2 (in steps of 0.02). The range of network sizes was chosen to reflect moderately well-sampled empirical webs while working within our computational limits, while the range of connectance values was chosen to cover that observed in most empirical food webs~\citep{Dunne2002e}. We generated a total of 100 networks with each combination of parameters, for a total of 6000 networks. All networks were generated using the function "nichemodel" within the Julia language package \emph{BioEnergeticFoodWebs}~\citep{bioenergeticfw,Delmas2017}. If a simulated network contained any disconnected species (species without predators or prey) or disconnected components (a group of species connected among themselves but not to the rest of the network), the network was rejected and a new network simulated. Finally, networks where the path lengths between each species and a basal resource could not be resolved (i.e., trophic levels were undefined) were rejected and new networks simulated.


	After generating the network structure, we simulated community dynamics using the function "simulate" from the Julia language package \emph{BioEnergeticFoodWebs}~\citep{bioenergeticfw,Delmas2017}. This function uses Lotka-Volterra predator-prey models including density dependence and type 2 functional responses for all species (please see~\citet{Delmas2017} for full details).
	All non-basal species were designated as vertebrates to ensure a good match between metabolic and predator-prey body-mass ratio values. Metabolic rates in the Lotka-Volterra model are based on each species' body mass (i.e. mass of a single individual). We assigned relative body masses based on each species' trophic level, which was, in turn, calculated based on the food-web structure provided by the niche model. After basal species were assigned a body mass of 1, we used a predator-prey body-mass ratio of 3.065 to calculate the relative body masses of higher trophic levels. We selected this ratio based on the estimate for vertebrates (averaged across ecosystem and metabolic types) in~\citet{Brose2006}. We excluded reported body-mass ratios for invertebrates as these could include parasites and parasitoids, which are generally smaller than their prey, and because interactions among vertebrates are better represented in the food-web literature than interactions involving invertebrates.
	
	
	The persistence of each species in our simulated networks also depends on its population biomass. 
	We randomly assigned initial population biomasses (i.e. cumulative biomass across all individuals of a species) for each species from a uniform distribution [0,1]. Note that population biomasses and individual body masses are not calculated on the same scale. We then simulated community dynamics for 1000 time steps to obtain an `equilibrium' community. To ensure that species did not `recover' from unrealistically low biomasses during the simulation, we considered a species extinct if it dropped below an arbitrary threshold biomass of 1$\times10^{-5}$. When simulating initial (i.e. pre-disturbance or equilibrium) dynamics, we rejected any network where one or more species dropped below this biomass threshold. Consumers were assumed to have no preferences such that the consumption rate $w_{ij}$ of predator $i$ eating prey $j$ is equal to $\frac{1}{n}$, where $n$ is the number of prey for predator $i$. If the network did not retain all species for 1000 time steps, a new set of initial population biomasses was applied and the simulation repeated.
	If a set of biomasses where all species persist still had not been reached after 100 sets of randomly-assigned initial biomasses, we discarded the network and simulated another to replace it.


\clearpage

\section*{S2: Testing consistency of times to extinction across removals}

	\subsection*{Methods}

		We simulated population dynamics within each food web after removing each species separately. 
		In order to assess species' overall vulnerability to the loss of an arbitrary other species within the food web, we used mean time to extinction across all removals as a response. 
		To ensure that this is a robust measure, we calculated the Pearson correlation of times to extinction for each species across all extinctions in each network. 
		We then tested whether the strength of these correlations varied with species richness and/or connectance by fitting a general linear model including fixed effects of species richness, connectance, and their interaction, as well as a random effect for network ID. 
		We fit the model using the R~\citep{R} function `lmer' from the package \emph{lmerTest}~\citep{lmerTest}.


	\subsection*{Results}

		In general, time to extinction was highly correlated across removals (Fig.~\ref{extorder_corrs}). %figure_creation/extinction_order_correlations.py
		The mean Pearson correlation for times to extinction across all removals within a network was 0.903 (range: 0.512-0.973). % stat_analysis/mean_correlation_extorder_tests.R
		This means that, in general, the species which go extinct fastest after species $i$ is removed also go extinct fastest after species $j$ is removed.
		The correlation was stronger in larger webs, particularly those with high connectance ($\beta_{S}$=1.33$\times10^-3$, $p$\textless0.001; $\beta_{C}$=-8.08$\times10^-2$, $p$\textless0.001; and $\beta_{S:C}$=2.12$\times10^-3$, $p$\textless0.001, respectively). 
		Mean time to extinction is, therefore, a good measure of a species' overall vulnerability.
		% [[Could look at which removals cause the biggest deviation from the mean. Are species which have particular roles/trophic levels causing particularly odd extinction timings?]]


		\begin{figure}[h!]
			\caption{\textbf{A)} Time to extinction for each species within a simulated network was highly correlated across removals. Circles indicate the mean correlation of time to extinction across removals for all species in all 100 simulated networks for a given combination of species richness and connectance. Lines indicate the predicted correlation based on the fixed effects of a linear model including species richness, connectance, and the interaction between them, as well as a random effect of network. Symbol and line colors indicate connectance. \textbf{B)} Mean time to extinction across all species within a network was slightly longer in small and less-connected networks. As in \textbf{A)}, line colors indicate connectance. \textbf{C)} Mean time to extinction was more strongly associated with connectance than species richness, with more-connected networks having shorter mean times to extinction. Line colors indicate species richness.}
			\label{extorder_corrs}
			\includegraphics[width=.75\textwidth]{figures/extinction_order/extorder_correlations.eps}
			\end{figure}		


\clearpage

\section*{S3: Details of PERMANOVA methods and results}

	\subsection*{Methods}


		To test for a relationship between time to extinction and species' overall roles, we fit a series of PERMANOVAs~\citep{Anderson2001} relating Bray-Curtis dissimilarity~\citep{Baker2015,Cirtwill2015} in species' raw motif roles to differences in mean time to extinction (Table 1, \emph{Main Text}).
		Due to computational limits, we were unable to fit a single PERMANOVA for all networks.
		To avoid effects of network size and connectance on time to extinction, we therefore fit separate PERMANOVAs for each combination of network size and connectance (60 PERMANOVAs in total).
		We fit all PERMANOVAs using the R~\citep{R} function 'adonis' from the package \emph{vegan}~\citep{vegan} and calculated $p$-values using 9999 unstratified permutations.
		As conducting so many tests risks obtaining significant results by chance, we applied the correlated Bonferroni correction~\citep{Drezner2016} before evaluating significance.
		
        
        In addition to participation in particular motifs affecting a species' vulnerability to extinction, it may also be that those species most vulnerable to extinction (i.e. short time to extinction) have more variable roles than those which are least vulnerable, or vice versa. 
        If there is unequal variance in roles across different times to extinction, this could cause false positive results in PERMANOVA tests.
        To account for this possibility, we first calculated the dispersion of roles for each value of mean time to extinction (treated categorically for this analysis), relative to the centroid for all species with the same mean time to extinction, using the function `betadisper' from the R~\citep{R} package \emph{vegan}~\citep{vegan}.
        We then tested whether some levels of mean time to extinction are associated with more widely-dispersed roles using an ANOVA test, fit using the function `anova' within the package \emph{vegan}~\citep{vegan}.
        Finally, to test whether role dispersion generally increases or decreases with increasing mean times to extinction, we fit a linear model relating role dispersion to mean time to extinction using the R~\citep{R} base function `lm'.


	\subsection*{Results}


		Taken individually, each PERMANOVA was significant (all $p$\textless0.025). Moreover, after applying the correlated Bonferroni correction~\citep{Drezner2016}, all PERMANOVAs remained significant.
		All PERMANOVAs were significant, indicating that species' overall motif roles are related to their mean times to extinction.
		However, the significant betadisper results suggest that some of these PERMANOVA results may be false positives.
		After applying the correlated Bonferroni correction~\citep{Drezner2016}, both the ANOVAs testing for non-homogeneous variability of roles and regressions testing for relationships between role variability and time to extinction remained significant.


		\begin{figure}[h!]
			\caption{Here we show (\textbf{A}) the pseudo-$F$ statistics and (\textbf{B}) the $p$-values for each PERMANOVA relating species' roles to their mean extinction order when all species in the web are separately removed. We fit one PERMANOVA per combination of species richness and connectance. $p$-values for each PERMANOVA are based on 9999 permutations, stratified by network. Symbols below the dotted line in \textbf{B} indicate a significant $p$-values. }
			\label{permfig}
			\includegraphics[height=.5\textheight]{figures/extinction_order/permanova_summary_paper_full.eps}
			\end{figure}


		\begin{table}[h!]
			\caption{For each combination of species richness (S) and connectance (C), the mean extinction order of a focal species was related to its raw motif role. We tested this using a series of PERMANOVAs with 9999 permutations each. Here we show the mean of correlations of extinction orders for a focal species across all removed species ($R^2$) and all 100 simulated networks for each combination of S and C, as well as the pseudo-$F$ statistic and $p$-value for each PERMANOVA. All tests remained significant after applying the correlated Bonferroni correction~\citep{Drezner2016}.}
			\label{permtable}
			\begin{tabular}{c c | c | c c ||c c | c | c c |}
				S	&	C	&	$R^2$	&	pseudo-$F$	&	$p$-value	&	S	&	C &	$R^2$	&	pseudo-$F$	&	$p$-value\\ 
				\hline
				50	&	0.02	&	0.789	&	86.7	&	0.017	&	80	&	0.02	&	0.866	&	107	&	0.013	\\
				50	&	0.04	&	0.813	&	66.4	&	0.013	&	80	&	0.04	&	0.898	&	114	&	0.014	\\
				50	&	0.06	&	0.845	&	70.2	&	0.014	&	80	&	0.06	&	0.9	&	128	&	0.016	\\  
				50	&	0.08	&	0.843	&	77.4	&	0.015	&	80	&	0.08	&	0.908	&	147	&	0.018	\\
				50	&	0.10	&	0.857	&	75.0	&	0.015	&	80	&	0.10	&	0.914	&	134	&	0.016	\\
				50	&	0.12	&	0.868	&	104	&	0.020	&	80	&	0.12	&	0.915	&	140	&	0.017	\\
				50	&	0.14	&	0.867	&	83.8	&	0.016	&	80	&	0.14	&	0.921	&	146	&	0.018	\\
				50	&	0.16	&	0.872	&	89.7	&	0.018	&	80	&	0.16	&	0.923	&	125	&	0.015	\\
				50	&	0.18	&	0.876	&	88.7	&	0.017	&	80	&	0.18	&	0.925	&	118	&	0.015	\\
				50	&	0.20	&	0.88	&	103	&	0.020	&	80	&	0.20	&	0.926	&	123	&	0.015	\\
				60	&	0.02	&	0.82	&	92.7	&	0.015	&	90	&	0.02	&	0.884	&	121	&	0.013	\\
				60	&	0.04	&	0.846	&	86.0	&	0.014	&	90	&	0.04	&	0.906	&	142	&	0.016	\\
				60	&	0.06	&	0.865	&	91.2	&	0.015	&	90	&	0.06	&	0.915	&	160	&	0.018	\\
				60	&	0.08	&	0.872	&	99.7	&	0.016	&	90	&	0.08	&	0.923	&	151	&	0.017	\\
				60	&	0.10	&	0.887	&	92.1	&	0.015	&	90	&	0.10	&	0.923	&	128	&	0.014	\\
				60	&	0.12	&	0.883	&	96.9	&	0.016	&	90	&	0.12	&	0.927	&	128	&	0.014	\\
				60	&	0.14	&	0.891	&	102	&	0.017	&	90	&	0.14	&	0.928	&	126	&	0.014	\\
				60	&	0.16	&	0.89	&	106	&	0.017	&	90	&	0.16	&	0.931	&	138	&	0.015	\\
				60	&	0.18	&	0.893	&	107	&	0.018	&	90	&	0.18	&	0.934	&	107	&	0.012	\\
				60	&	0.20	&	0.899	&	137	&	0.022	&	90	&	0.20	&	0.936	&	127	&	0.014	\\
				70	&	0.02	&	0.848	&	91.4	&	0.013	&	100	&	0.02	&	0.899	&	125	&	0.012	\\
				70	&	0.04	&	0.875	&	108	&	0.015	&	100	&	0.04	&	0.917	&	191	&	0.019	\\
				70	&	0.06	&	0.877	&	111	&	0.016	&	100	&	0.06	&	0.923	&	206	&	0.020	\\
				70	&	0.08	&	0.898	&	112	&	0.016	&	100	&	0.08	&	0.932	&	176	&	0.017	\\
				70	&	0.10	&	0.904	&	134	&	0.019	&	100	&	0.10	&	0.934	&	148	&	0.015	\\
				70	&	0.12	&	0.907	&	124	&	0.017	&	100	&	0.12	&	0.934	&	156	&	0.015	\\
				70	&	0.14	&	0.906	&	118	&	0.017	&	100	&	0.14	&	0.939	&	98.3	&	0.010	\\
				70	&	0.16	&	0.909	&	122	&	0.017	&	100	&	0.16	&	0.938	&	144	&	0.014	\\
				70	&	0.18	&	0.913	&	99.9	&	0.014	&	100	&	0.18	&	0.939	&	118	&	0.012	\\
				70	&	0.20	&	0.917	&	122	&	0.017	&	100	&	0.20	&	0.942	&	105	&	0.010	\\
				\hline
			\end{tabular}
			\end{table}


		\begin{table}[h!]
			\caption{For each combination of species richness (S) and connectance (C), some levels of mean time to extinction were associated with more variable roles than others. This may cause false positives in the PERMANOVAs reported in table~\ref{permtable}. In all cases, the variability of species' roles increased significantly with increasingly long mean times to extinction. All tests remained significant after applying the correlated Bonferroni correction~\citep{Drezner2016}.}
			\label{disptable}
			\footnotesize
			\begin{tabular}{c c | c c| c c ||c c | c c | c c |}
				&		&	\multicolumn{2}{c|}{ANOVA}	&\multicolumn{2}{c||}{Regression} 			& & & 	\multicolumn{2}{c|}{ANOVA}		 	&	 \multicolumn{2}{c|}{Regression} 			\\
	            S	&	C	&	$F$	&	$p$-value	&	$\beta$	&	$p$-value	&	S	&	C	&	$F$	&	$p$-value	&	$\beta$	&	$p$-value	\\
				\hline
	        50	&	0.02	&	3.38	&	\textless0.001	&	3.36$\times10^{-4}$	&	\textless0.001	&	80	&	0.02	&	3.72	&	\textless0.001	&	1.74$\times10^{-4}$	&	\textless0.001	\\
	        50	&	0.04	&	3.90	&	\textless0.001	&	2.60$\times10^{-4}$	&	\textless0.001	&	80	&	0.04	&	3.87	&	\textless0.001	&	1.34$\times10^{-4}$	&	\textless0.001	\\
	        50	&	0.06	&	3.75	&	\textless0.001	&	2.12$\times10^{-4}$	&	\textless0.001	&	80	&	0.06	&	3.86	&	\textless0.001	&	1.15$\times10^{-4}$	&	\textless0.001	\\
	        50	&	0.08	&	3.59	&	\textless0.001	&	1.81$\times10^{-4}$	&	\textless0.001	&	80	&	0.08	&	3.83	&	\textless0.001	&	1.04$\times10^{-4}$	&	\textless0.001	\\
	        50	&	0.10	&	3.56	&	\textless0.001	&	1.76$\times10^{-4}$	&	\textless0.001	&	80	&	0.10	&	3.59	&	\textless0.001	&	8.39$\times10^{-5}$	&	\textless0.001	\\
	        50	&	0.12	&	3.76	&	\textless0.001	&	1.60$\times10^{-4}$	&	\textless0.001	&	80	&	0.12	&	3.49	&	\textless0.001	&	8.00$\times10^{-5}$	&	\textless0.001	\\
	        50	&	0.14	&	3.39	&	\textless0.001	&	1.52$\times10^{-4}$	&	\textless0.001	&	80	&	0.14	&	3.27	&	\textless0.001	&	7.80$\times10^{-5}$	&	\textless0.001	\\
	        50	&	0.16	&	3.34	&	\textless0.001	&	1.34$\times10^{-4}$	&	\textless0.001	&	80	&	0.16	&	3.02	&	\textless0.001	&	6.34$\times10^{-5}$	&	\textless0.001	\\
	        50	&	0.18	&	3.23	&	\textless0.001	&	1.28$\times10^{-4}$	&	\textless0.001	&	80	&	0.18	&	2.72	&	\textless0.001	&	5.46$\times10^{-5}$	&	\textless0.001	\\
	        50	&	0.20	&	3.12	&	\textless0.001	&	1.24$\times10^{-4}$	&	\textless0.001	&	80	&	0.20	&	2.55	&	\textless0.001	&	4.11$\times10^{-5}$	&	\textless0.001	\\
	        60	&	0.02	&	3.58	&	\textless0.001	&	2.78$\times10^{-4}$	&	\textless0.001	&	90	&	0.02	&	3.56	&	\textless0.001	&	1.52$\times10^{-4}$	&	\textless0.001	\\
	        60	&	0.04	&	3.88	&	\textless0.001	&	2.05$\times10^{-4}$	&	\textless0.001	&	90	&	0.04	&	4.00	&	\textless0.001	&	1.13$\times10^{-4}$	&	\textless0.001	\\
	        60	&	0.06	&	3.72	&	\textless0.001	&	1.74$\times10^{-4}$	&	\textless0.001	&	90	&	0.06	&	3.64	&	\textless0.001	&	9.23$\times10^{-5}$	&	\textless0.001	\\
	        60	&	0.08	&	3.47	&	\textless0.001	&	1.50$\times10^{-4}$	&	\textless0.001	&	90	&	0.08	&	3.44	&	\textless0.001	&	7.63$\times10^{-5}$	&	\textless0.001	\\
	        60	&	0.10	&	3.58	&	\textless0.001	&	1.23$\times10^{-4}$	&	\textless0.001	&	90	&	0.10	&	3.24	&	\textless0.001	&	7.03$\times10^{-5}$	&	\textless0.001	\\
	        60	&	0.12	&	3.45	&	\textless0.001	&	1.29$\times10^{-4}$	&	\textless0.001	&	90	&	0.12	&	3.02	&	\textless0.001	&	5.14$\times10^{-5}$	&	\textless0.001	\\
	        60	&	0.14	&	3.45	&	\textless0.001	&	1.12$\times10^{-4}$	&	\textless0.001	&	90	&	0.14	&	2.85	&	\textless0.001	&	5.56$\times10^{-5}$	&	\textless0.001	\\
	        60	&	0.16	&	3.30	&	\textless0.001	&	1.11$\times10^{-4}$	&	\textless0.001	&	90	&	0.16	&	2.92	&	\textless0.001	&	5.26$\times10^{-5}$	&	\textless0.001	\\
	        60	&	0.18	&	3.07	&	\textless0.001	&	1.03$\times10^{-4}$	&	\textless0.001	&	90	&	0.18	&	2.58	&	\textless0.001	&	4.11$\times10^{-5}$	&	\textless0.001	\\
	        60	&	0.20	&	3.15	&	\textless0.001	&	9.84$\times10^{-5}$	&	\textless0.001	&	90	&	0.20	&	2.59	&	\textless0.001	&	4.17$\times10^{-5}$	&	\textless0.001	\\
	        70	&	0.02	&	3.45	&	\textless0.001	&	2.16$\times10^{-4}$	&	\textless0.001	&	100	&	0.02	&	3.99	&	\textless0.001	&	1.31$\times10^{-4}$	&	\textless0.001	\\
	        70	&	0.04	&	3.76	&	\textless0.001	&	1.64$\times10^{-4}$	&	\textless0.001	&	100	&	0.04	&	3.95	&	\textless0.001	&	9.81$\times10^{-5}$	&	\textless0.001	\\
	        70	&	0.06	&	3.83	&	\textless0.001	&	1.46$\times10^{-4}$	&	\textless0.001	&	100	&	0.06	&	3.65	&	\textless0.001	&	8.23$\times10^{-5}$	&	\textless0.001	\\
	        70	&	0.08	&	3.64	&	\textless0.001	&	1.11$\times10^{-4}$	&	\textless0.001	&	100	&	0.08	&	3.31	&	\textless0.001	&	6.52$\times10^{-5}$	&	\textless0.001	\\
	        70	&	0.10	&	3.39	&	\textless0.001	&	9.72$\times10^{-5}$	&	\textless0.001	&	100	&	0.10	&	3.11	&	\textless0.001	&	5.62$\times10^{-5}$	&	\textless0.001	\\
	        70	&	0.12	&	3.42	&	\textless0.001	&	9.64$\times10^{-5}$	&	\textless0.001	&	100	&	0.12	&	3.05	&	\textless0.001	&	5.07$\times10^{-5}$	&	\textless0.001	\\
	        70	&	0.14	&	3.09	&	\textless0.001	&	8.62$\times10^{-5}$	&	\textless0.001	&	100	&	0.14	&	2.54	&	\textless0.001	&	3.22$\times10^{-5}$	&	\textless0.001	\\
	        70	&	0.16	&	3.05	&	\textless0.001	&	8.31$\times10^{-5}$	&	\textless0.001	&	100	&	0.16	&	2.56	&	\textless0.001	&	3.44$\times10^{-5}$	&	\textless0.001	\\
	        70	&	0.18	&	2.85	&	\textless0.001	&	6.25$\times10^{-5}$	&	\textless0.001	&	100	&	0.18	&	2.58	&	\textless0.001	&	3.63$\times10^{-5}$	&	\textless0.001	\\
	        70	&	0.20	&	2.81	&	\textless0.001	&	6.77$\times10^{-5}$	&	\textless0.001	&	100	&	0.20	&	2.32	&	\textless0.001	&	2.17$\times10^{-5}$	&	\textless0.001	\\

	    \end{tabular}
	    \end{table}
    
\clearpage

\section*{S4: Details of PLSR methods}

	The PERMANOVAs indicate whether roles as a whole are related to mean time to extinction but do not reveal which motifs have the strongest effect on time to extinction.
	To answer this question, we used a set of partial least squares (PLS) regressions to identify combinations of motifs which, together, explain substantial variation in time to extinction. 
	Similar to a principal components analysis (PCA), PLS projects the observed variables (i.e., participation in different motifs) into a new space and identifies latent variables made up of linear combinations of the observed variables~\citep{Mevik2004,pls} which, together, explain substantial variation in mean time to extinction.
	
	
	First, we used mean time to extinction as the response and raw motif roles as well as network size, connectance, the interaction between size and connectance, in-degree (number of prey), and shortest trophic level (STL)~\citep{Hairston1993} as predictors (Table~\ref{overview_table}).
	We include additional measures of network structure and species roles as these may affect the motif roles available to each focal species.
	Next, to test whether any relationship between raw motif roles and time to extinction might be due differences in the roles of species appearing in different total numbers of motifs, we fit a second PLS regression using degree-normalized roles instead of the raw roles, with all other variables identical to the first regression.
	Third, to understand whether it is the absolute frequency of motifs or the relative frequency compared to other species in the network that is related to time to extinction, we fit a final PLS regression using network-normalized roles as a predictor and all other variables identical to the first regression.
	
	
	We fit all regressions using the R~\citep{R} function 'plsr' from the package \emph{pls}~\citep{pls}.
	To prevent differences in range and intercept values from influencing the fit of the PLS model, we centered and scaled all variables.
	After initial fitting, we cross-validated each regression using 10 randomly-selected segments of the data and re-fitting the regression and then calculated the mean squared error of prediction (MSEP) for each model.
	MSEP is a measure of the error obtained when re-fitting a PLS or PCA model on test data, and is commonly used to select the optimum number of components~\citep{Mevik2004}.
	In order to balance obtaining a low MSEP with identifying a parsimonious model, we defined the optimum model as that with the fewest components that nevertheless had an MSEP within one standard deviation of the lowest MSEP obtained for any model.
	Model selection was performed using the R~\citep{R} function 'selectNcomp' from the package \emph{plsr}~\citep{pls} using the method `onesigma'.
	After selecting the optimum number of components for each model, we re-fit the PLS regression including only the selected components. 
	We then summed the coefficients of each predictor across axes to obtain an overall measure of the effect of each predictor on mean time to extinction.

\clearpage


\section*{S5: Relating motif roles to other measures}

    \begin{table}[h!]
        \caption{Slopes and $p$-values for linear regressions relating degree (number of interaction partners) to the raw counts of motifs, degree-normalized proportions of motifs, and network-normalized $Z$-scores of motifs. Motifs are named as in~\citet{Stouffer2007}. Motifs S1 (\emph{three-species chain}), S2 (\emph{omnivory}), S4 (\emph{direct competition}), and S5 (\emph{apparent competition}) are the 'stable' motifs of particular interest in our study (highlighted in bold). All regressions remained significant after applying the correlated Bonferroni correction~\citep{Drezner2016}.}
        \label{degree_lms}
        \begin{tabular}{l | c c c c c c }
        \multirow{2}{*}{Motif} & \multicolumn{2}{c}{Raw roles} & \multicolumn{2}{c}{Degree-normalized roles} & \multicolumn{2}{c}{Network-normalized roles} \\
        & Slope & $p$-value & Slope & $p$-value & Slope & $p$-value \\
        \hline
        \textbf{S1}	&	6.17	&	\textbf{\textless0.001}	& -1.24$\times10^{-3}$	&	\textbf{\textless0.001}	&	2.61$\times10^{-2}$	&	\textbf{\textless0.001}	\\
        \textbf{S2}	&	11.1	&	\textbf{\textless0.001}	& 4.21$\times10^{-3}$	&	\textbf{\textless0.001}	&	3.73$\times10^{-2}$	&	\textbf{\textless0.001}	\\
        \textbf{S4}	&	3.18	&	\textbf{\textless0.001}	& -1.04$\times10^{-3}$	&	\textbf{\textless0.001}	&	1.63$\times10^{-2}$	&	\textbf{\textless0.001}	\\
        \textbf{S5}	&	12.6	&	\textbf{\textless0.001}	& -4.13$\times10^{-3}$	&	\textbf{\textless0.001}	&	3.02$\times10^{-2}$	&	\textbf{\textless0.001}	\\
        \hline
        S3	&	6.73$\times10^{-4}$	&	\textbf{\textless0.001}	&	2.96$\times10^{-7}$	&	\textbf{\textless0.001}	&	8.90$\times10^{-4}$	&	\textbf{\textless0.001}	\\
        D1	&	2.21	&	\textbf{\textless0.001}	&	8.83$\times10^{-4}$	&	\textbf{\textless0.001}	&	2.52$\times10^{-2}$	&	\textbf{\textless0.001}	\\
        D2	&	0.797	&	\textbf{\textless0.001}	&	3.98$\times10^{-4}$	&	\textbf{\textless0.001}	&	2.11$\times10^{-2}$
        	&	\textbf{\textless0.001}	\\
        D3	&	1.78	&	\textbf{\textless0.001}	&	6.37$\times10^{-4}$	&	\textbf{\textless0.001}	&	2.08$\times10^{-2}$	&	\textbf{\textless0.001}	\\
        D4	&	0.203	&	\textbf{\textless0.001}	&	8.98$\times10^{-5}$	&	\textbf{\textless0.001}	&	1.22$\times10^{-2}$	&	\textbf{\textless0.001}	\\
        D5	&	3.77$\times10^{-2}$	&	\textbf{\textless0.001}	&	1.52$\times10^{-5}$	&	\textbf{\textless0.001}	&	1.11$\times10^{-2}$	&	\textbf{\textless0.001}	\\
        D6	&	0.127	&	\textbf{\textless0.001}	&	5.85$\times10^{-5}$	&	\textbf{\textless0.001}	&	1.78$\times10^{-2}$	&	\textbf{\textless0.001}	\\
        D7	&	0.256	&	\textbf{\textless0.001}	&	1.15$\times10^{-4}$	&	\textbf{\textless0.001}	&	1.94$\times10^{-2}$	&	\textbf{\textless0.001}	\\
        D8	&	4.04$\times10^{-2}$	&	\textbf{\textless0.001}	&	1.61$\times10^{-5}$	&	\textbf{\textless0.001}	&	1.06$\times10^{-2}$	&	\textbf{\textless0.001}	\\
        \hline
        \end{tabular}
        \end{table}


    \begin{table}[h!]
        \caption{Slopes and $p$-values for linear regressions relating trophic level (STL; height in the food web) to the raw counts of motifs, degree-normalized proportions of motifs, and network-normalized $Z$-scores of motifs. Motifs are named as in~\citet{Stouffer2007}. Motifs S1 (\emph{three-species chain}), S2 (\emph{omnivory}), S4 (\emph{direct competition}), and S5 (\emph{apparent competition}) are the 'stable' motifs of particular interest in our study (highlighted in bold). All regressions remained significant after applying the correlated Bonferroni correction~\citep{Drezner2016}.}
        \label{TL_lms}
        \begin{tabular}{l | c c c c c c }
        \multirow{2}{*}{Motif} & \multicolumn{2}{c}{Raw roles} & \multicolumn{2}{c}{Degree-normalized roles} & \multicolumn{2}{c}{Network-normalized roles} \\
        & Slope & $p$-value & Slope & $p$-value & Slope & $p$-value \\
        \hline
        \textbf{S1}	&	9.58	&	\textbf{\textless0.001}	&	-8.32$\times10^{-3}$	&	\textbf{\textless0.001}	&	-0.156	&	\textbf{\textless0.001}	\\
        \textbf{S2}	&	3.36	&	\textbf{\textless0.001}	&	3.62$\times10^{-3}$	&	\textbf{\textless0.001}	&	-0.194	&	\textbf{\textless0.001}	\\
        \textbf{S4}	&	3.48	&	\textbf{\textless0.001}	&	-4.59$\times10^{-3}$	&	\textbf{\textless0.001}	&	-9.51$\times10^{-2}$	&	\textbf{\textless0.001}	\\
        \textbf{S5}	&	24.0	&	\textbf{\textless0.001}	&	-6.35$\times10^{-3}$	&	\textbf{\textless0.001}	&	-9.66$\times10^{-2}$	&	\textbf{\textless0.001}	\\
        \hline													
        S3	&	2.96$\times10^{-3}$	&	\textbf{\textless0.001}	&	4.82$\times10^{-6}$	&	\textbf{\textless0.001}	&	2.91$\times10^{-3}$	&	\textbf{\textless0.001}	\\
        D1	&	2.48	&	\textbf{\textless0.001}	&	4.50$\times10^{-3}$	&	\textbf{\textless0.001}	&	-4.16$\times10^{-2}$	&	\textbf{\textless0.001}	\\
        D2	&	1.09	&	\textbf{\textless0.001}	&	1.61$\times10^{-3}$	&	\textbf{\textless0.001}	&	-4.53$\times10^{-2}$	&	\textbf{\textless0.001}	\\
        D3	&	3.72	&	\textbf{\textless0.001}	&	6.89$\times10^{-3}$	&	\textbf{\textless0.001}	&	-1.37$\times10^{-2}$	&	\textbf{\textless0.001}	\\
        D4	&	0.804	&	\textbf{\textless0.001}	&	1.34$\times10^{-3}$	&	\textbf{\textless0.001}	&	2.82$\times10^{-2}$	&	\textbf{\textless0.001}	\\
        D5	&	0.112	&	\textbf{\textless0.001}	&	2.05$\times10^{-4}$	&	\textbf{\textless0.001}	&	1.85$\times10^{-2}$	&	\textbf{\textless0.001}	\\
        D6	&	9.93$\times10^{-2}$	&	\textbf{\textless0.001}	&	1.86$\times10^{-4}$	&	\textbf{\textless0.001}	&	-2.65$\times10^{-2}$	&	\textbf{\textless0.001}	\\
        D7	&	0.380	&	\textbf{\textless0.001}	&	6.64$\times10^{-4}$	&	\textbf{\textless0.001}	&	-8.81$\times10^{-3}$	&	\textbf{\textless0.001}	\\
        D8	&	0.135	&	\textbf{\textless0.001}	&	2.46$\times10^{-4}$	&	\textbf{\textless0.001}	&	2.10$\times10^{-2}$	&	\textbf{\textless0.001}	\\
        \hline
        \end{tabular}
        \end{table}
        

\clearpage

\section*{S6: Motif labels}

	\begin{figure}[h!]
		\caption{There are 13 unique three-species motifs which can appear in food webs. Motifs S1, S2, S4, and S5 have been identified as more stable than other motifs when modeled in isolation and as potentially increasing the stability of empirical food webs. Labels are as in~\citet{Stouffer2007}.}
		\label{motifs}
		\includegraphics[width=.8\textwidth]{figures/motifs.eps}
		\end{figure}

\end{spacing}
\clearpage

    \bibliographystyle{ecollett} 
    \bibliography{MyCollection} % Abbreviate journal titles.


\end{document}



