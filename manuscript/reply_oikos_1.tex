\documentclass[12pt]{letter}

% \usepackage[britdate]{SU-letter}
\usepackage[britdate]{UH-letter}
\usepackage{times}
\usepackage{letterbib}
\usepackage{geometry}
\usepackage[round]{natbib}
\usepackage{graphicx}
\geometry{a4paper}
\usepackage[T1]{fontenc}
\usepackage[utf8]{inputenc}
\usepackage{authblk}
\usepackage[running]{lineno}
\usepackage{amsmath,amsfonts,amssymb}
% \usepackage[margin=10pt,font=small,labelfont=bf]{caption}

%\usepackage{natbib}
% \bibpunct[; ]{(}{)}{;}{a}{,}{;}

\newenvironment{refquote}{\bigskip \begin{it}}{\end{it}\smallskip}

\newenvironment{figure}{}

\position{Postdoctoral fellow}
\department{Department of Agricultural Sciences}
\location{University of Helsinki}
\cityzip{Helsinki, Finland}
\telephone{}
\fax{}
\email{alyssa.cirtwill@helsinki.fi}
\url{http://cirtwill.github.io}
\name{Dr. Alyssa R. Cirtwill}

\begin{document}

\begin{letter}{\bf Prof. Dries Bonte [[[CHECK SPEC CHARS]]]\\
Oikos Editorial Office \\
Lund University \\
Lund, Sweden}


\opening{Dear Prof. Bonte:}

%   My co-author and I are happy to invite you to consider our manuscript
%   \emph{Stable motifs delay species loss in simulated food webs} for publication in \textbf{Oikos}.


%   Motifs --unique configurations of small numbers of interacting species-- are becoming more common as a tool for describing the meso-scale structure of ecological networks (i.e., a species 'neighborhood' within the network, including direct and close indirect interactions). 
%   The set of motifs a species participates in has already been linked to species' traits, taxonomy, and location.
%   More recent research has also suggested that some 'stable' motifs, in isolation, dampen disturbances and facilitate persistence of the species within them while others do not.
%   This suggests that species which participate in many of these `stable' motifs should be more likely to persist following a disturbance than species which participate in more of the `unstable' motifs.
%   If this is the case, motifs offer the possibility of estimating extinction risk within all types of ecological communities, without necessarily having to know the structure of the entire network.


%   Here we test this possibility using a large set of realistic simulated networks. In brief, we find that participating in more of the stable motifs is generally associated with longer persistence. Participating more often in the omnivory motif, however, is associated with shorter persistence. Motifs can therefore be added to the toolbox of researchers wishing to estimate species' risk of extinction. In particular, motifs offer a more detailed perspective on network structure than simpler measures such as degree (number of interaction partners) or trophic level and may be useful when comparing the risk of apparently similar taxa.


%   Our results also offer a potential explanation for the variable effects of omnivory on community stability. In our networks, participating more often in the omnivory motif was associated with greater extinction risk. However, species with high numbers of interaction partners (which tend to have lower extinction risk) appeared more often in the omnivory motif than species with few interaction partners. The apparent effect of omnivory may therefore depend on the balance between the effect of omnivory \emph{per se} and the effect of numbers of interaction partners.


%   Predicting which species are most likely to go extinct following a disturbance is one of the pressing ecological tasks of our time. 
%   Motifs can contribute to solving this problem as they provide a rich description of the ways species fit into their communities which can be applied to any ecological community and, unlike whole-network measures of stability, can be calculated for a focal species or group even if the structure of an entire community is unknown.
%   Moreover, our results advance the basic science of network ecology by establishing the relationships between motif participation, degree, and trophic level. 
%   Because of the broad applicability and combination of theoretical and practical relevance of our results, we believe that \textbf{Oikos} is the best place to publish this manuscript. 
%   We hope that you will agree and eagerly await your reply.


%   A non-peer reviewed preprint of this manuscript has been posted to bioRxiv \\\noindent(doi: https://www.biorxiv.org/content/10.1101/2021.04.06.438635).

% \closing{Best regards,}

01-Jul-2021

Dear Dr. Cirtwill:

Manuscript ID OIK-08610 entitled "Stable motifs delay species loss in simulated food webs" which you submitted to Oikos, has been reviewed.  The comments of the reviewer(s) and the recommendation by the SE are included at the bottom of this letter.

Based on the referee(s) comments, the subject editor has recommended substantial revisions to your manuscript.  Therefore, I invite you to respond to the reviewer(s)' comments and revise your manuscript.

At this stage, we strongly encourage authors to contribute with text (maximum 200 words each) for the sessions Speculations and Alternative Viewpoints (among co-authors) to appear after the Discussion session.  These sessions are not compulsory and aim at increasing the visibility of your published work and encourage debate in ecology. Contributions to these sessions can be after the manuscript undergoes peer review or just in the version before final acceptance for publication.   For detailed information go to: http://www.oikosjournal.org/authors/author-guidelines

Author Contribution Indication
The contributions of each author to this work must now be indicated when you submit your revised manuscript. To add Author Contributions using CRediT taxonomy (http://credit.niso.org/contributor-roles-defined/), simply click the “Provide CRediT Contribution” link for each author in the ‘Authors & Institutions’ step of the submission process. From there, you will be able to check applicable Author/Contributor Roles and, if available, specify the Degree of Contribution. You MUST provide this information as part of the revision process. Author Contributions will be published with the accepted article and cannot be edited after article acceptance. Therefore you must ensure the Author Contribution information you provide is accurate prior to final acceptance.

There are two ways to submit your revised manuscript. You may use the link below to submit your revision online with no need to enter log in details:

*** PLEASE NOTE: This is a two-step process. After clicking on the link, you will be directed to a webpage to confirm. ***

https://mc.manuscriptcentral.com/oikos?URL_MASK=745eb08638f94600bc8791bf54bdd55a

Alternatively log into https://mc.manuscriptcentral.com/oikos and enter your Author Center. You can use the revision link or you will find your manuscript title listed under "Manuscripts with Decisions."  Under "Actions," click on "Create a Revision."  Your manuscript number has been appended to denote a revision.  Please DO NOT upload your revised manuscripts as a new submission.

You will be unable to make your revisions directly on the original uploaded version of the manuscript.  Instead, revise your manuscript using a word processing program and save it on your computer.

Please highlight the changes to your manuscript within the document by using the track changes mode in MS Word or by using bold or colored text. Once the revised manuscript is prepared, you can upload it and submit it through your Author Center.

Please revise the manuscript carefully in line with suggestions made by the reviewer(s) and respond to the criticism in the space provided or as a separate uploaded cover letter.  In order to expedite the processing of the revised manuscript, please be as specific as possible in your response to the reviewer(s). Please, upload a file with your response to the reviewers' comments when submitting your revised version (make sure that the file is also anonymous)

IMPORTANT:  Your original files are available to you when you upload your revised manuscript.  Please delete any redundant files before completing the submission.

TEXT: Please read the instruction for authors thoroughly before resubmitting your manuscript.

FIGURES: Please ensure that your illustrations are of high quality (> 300 dpi), preferably as high resolution jpeg, pdf or gifs. Please avoid pixelated symbols and legends. The normal width of an illustration in Oikos is 7.5 cm which means that all legends and symbols should be easily readable at this size. When scientifically motivated, a larger width, 11 cm, is also available.

REFERENCES: Please revise the reference list so that it complies with our style as seen in current issues of Oikos.  Papers with incomplete or malformatted reference lists, will not be officially accepted for publication.

MANUSCRIPT DETAILS: If any changes have been made to the title and/or abstract, please ensure that the relevant parts in the manuscript details within your Author Center are updated accordingly.

Because we are trying to facilitate timely publication of manuscripts submitted to Oikos, your revised manuscript should be uploaded as soon as possible and not later than within 1 month from today.  If it is not possible for you to submit your revision in a reasonable amount of time, we may have to consider your paper as a new submission.  If you feel that you will be unable to submit your revision within the time allowed please contact me to discuss the possibility of extending the revision time.

PLEASE: Submit your revised manuscript as MS Word or OpenOffice file(s). Submit your manuscript as one complete file, which includes: the text, legends, illustrations, tables and references.

Also submit, as separate files, all illustrations in high resolution (see above).

Once again, thank you for submitting your manuscript to Oikos and I look forward to receiving your revision.

Wiley Editing Services Available to All Authors
Should you be interested, Wiley Editing Services offers expert help with manuscript, language, and format editing, along with other article preparation services. You can learn more about this service option at www.wileyauthors.com/eeo/preparation. You can also check out Wiley’s collection of free article preparation resources for general guidance about writing and preparing your manuscript at www.wileyauthors.com/eeo/prepresources.

Sincerely,

Prof. Dries Bonte
Editor-in-Chief, Oikos




\clearpage

\Large{\textbf{Reply to Subject Editor}}

    
    \begin{quotation}    
  Recommendation by the Subject Editor (Dr. François Munoz):

  The reviewers and me agree that the topic of the paper is interesting and fits well the scope of Oikos.
  Understanding how the participation of species to trophic network motifs relates to species extinction risk should convey novel theoretical insights on the dynamics of trophic interaction networks.

  As Reviewer 2 noticed, I found that the methodology is quite complex, regarding the simulation of networks and their dynamics, and also regarding the statistical analyses of the link between network-based metrics and species extinction risk.
  Therefore, I recommend that the authors carefully revise the manuscript in the light of the comments and suggestions from both reviewers. The revision works should allow simplifying the presentation of results and thereby make them easier to understand.

  I look forward receiving a revised version of your work.

  Sincerely,

  François
  \end{quotation}
  
  We have taken to heart the point which you and Reviewer 2 have raised about the complexity of our methodology. We have substantially revised our statistical analyses to better focus on our key questions: 1) are species' motif roles related to their risk of extinction? and 2) if so, which motifs show the strongest relationships? We have also revised the text to more clearly explain our simulation methodology. We hope that this revision is clearer and easier to follow.

\clearpage

\Large{\textbf{Reply to Reviewer 1}} [still needs preface.]

  In this work the author present a modelling study to explore how a species' participation in network motifs that are either stable or unstable relates to their extinction risk following a perturbation. The authors also discuss how species roles relate to other common network properties like species' degree and trophic position. 

  My only major wish is that the code was made available either as a supplement, or in a repository such as Zenodo or Figshare. The methods were fairly clear, but being able to review the code would be very helpful.


  \textbf{R:}  
  We could do that. I'm not entirely sure which version of the code was used for this but I can clean it up if you let me know where it is.


  \textbf{Minor comments:}

  1. Resolution of time to extinction [done]


    Why is mean time to extinction computed by simulating multiple rounds of 10 timesteps rather than simulating the 500 timesteps and identifying where extinctions occur? Does this mean that time to extinction is defined as number of rounds until extinction, or by counting the timesteps within each round until the biomass falls below 10^-5? 

    \textbf{R:} This was done due to computational limits when running the models and storing output. We felt that a robust sample size and reasonably broad range of network structures were more important than obtaining very precise measures of time to extinction (especially since we do not simulate any particular species of interest). Time to extinction is defined as number of rounds until extinction - we have revised the text to make this clearer. The revised section now reads:

    \begin{quotation}

            Once we identified species' roles in the equilibrium networks, we perturbed the networks by removing a single species. 
            We then simulated post-removal community dynamics for 500 time-steps.
            After every 10 time-steps, any species with a biomass below our threshold of 1$\times10^{-5}$ was considered to have gone extinct and its biomass was set to 0.
            This reduced temporal resolution was chosen due to computational limitations and allowed us to maintain a large set of simulated networks.
            For each species other than the artificially removed species, we defined the time to extinction as the round of 10 time-steps in which the species' biomass fell below our threshold. 

    \end{quotation}


  2. Role dispersion not clear [done?]


    I am not clear on what role dispersion (line 146) is in this case. Is this the variance of the counts for each motif per species? In other words, what does it mean to have a more or less variable role (line 208)?

    \textbf{R:} Dispersion is, essentially, a multivariate extension of variance which takes counts of all motifs into account simultaneously. 
    Each species has only one role; we are calculating the dispersion of roles among species with different mean times to extinction. We find that there is a greater variety of roles among species with long times to extinction than short times to extinction. 


    The relationship between city coordinates and temperature may be a useful analogy. Temparate cities are very widely dispersed in latitude and longitude, while very hot cities have lower dispersion as they are restricted to the tropics.


    We have revised this paragraph to be clearer that dispersion is a measure across species rather than within a single species' role and hope that this makes our methodology clearer.


    \begin{quotation}

            As non-homogeneous dispersions within groups (i.e., differences in the variability of roles across species) can cause false-positive PERMANOVA results~\citep{Anderson2001}, we used a series of ANOVAs to test the homogeneity of dispersion among roles associated with each mean time to extinction (using the R~\citep{R} package \emph{vegan}~\citep{vegan}; \emph{Appendix S3}).
            Finally, we tested whether long or short mean times to extinction were associated with more or less variable roles  using a linear model (fit using the R~\citep{R} base function `lm').

    \end{quotation}    


  3. Request for explanation of MSEP in main text [done]

    line 172: MSEP is defined in the Appendix but not the main text. I think it would be especially helpful if the text from lines 133-139 describing MSEP and providing a bit more detail on model selection was moved to the main text.

    \textbf{R:} We have expanded the description of MSEP in the main text as suggested.

    \begin{quotation}

          We fit all regressions using the R~\citep{R} function 'plsr' from the package \emph{pls}~\citep{pls} using centered and scaled variables and cross-validating each regression using 10 randomly-selected segments of the data.
          To define the optimum number of components, we used MSEP: a measure of error obtained by re-fitting a PLS or PCA model on test data~\citep{Mevik2004}.
          In order to balance obtaining a low MSEP with identifying a parsimonious model, we defined the optimum number of components for each regression as the lowest number of components that yielded an MSEP~\citep{Mevik2004} within one standard deviation of the lowest MSEP obtained for any model.

    \end{quotation}

  4. Confusion over de-scaled version of Figure 2 [done]

    line 178: Is the de-scaled version of the model presented in Figure 2 obtained by re-fitting the PLS using the raw values rather than centered and scaled values? If so I think you should explicitly mention this in the methods.   


    \textbf{R:} The de-scaled coefficients in Figure 2 refer to the same model as the scaled version. The de-scaled values are obtained by multiplying the coefficients fit to centered and scaled values by the standard deviation of the data (i.e., the scaling factor). We have revised the figure caption and panel labels to explicitly state that both scaled and de-scaled values refer to the same model, and hope that this figure is now clearer.


  5. Request for discussion of low explanatory power [to do at end]

    lines 217, 232, 246: Seems like the predictor variables are not explaining much of the variation in mean time to extinction (around 20-25\%), which is not necessarily bad, but what do you think might be missing? Could be worth including in the discussion.

    \textbf{R:}
    Position in motif? Link strengths? Could also be effects of removed species, which have been shown to be large.


  6. Corrections to table references [done]

    line 206: I think you mean to refer to Table S2 in Appendix S3.

    Appendix line 77 is missing a Table reference "Table ??"

    \textbf{R:} Both table references have been corrected. We thank the Reviewer for pointing out the errors.


\clearpage

\Large{\textbf{Reply to Reviewer 2}}

  \begin{quotation}
  Authors investigate the link between the motif roles of a species, i.e. its frequency of participation to each of the 13 triangular motifs inside a trophic network, and the time before it goes extinct after the primary extinction of another species in the network. Thus, they aim at identifying which motif roles contribute to a species stability or instability given its position in the network. They also try to disentangle the specific effect of the species motif roles from network level effects (richness, connectance) and other species level effects (trophic level, degree).


  This authors treat a very interesting and original question that is well grounded on earlier work in the literature, and which I think may help broaden our theoretical understanding of trophic dynamics. The authors made a noticeable effort in their simulation design to obtain realistic and stable body-mass structured vertebrate food webs.
  % Well that sounds positive!


  (now speakng directly in second person to the authors) However, I don’t understand why you base your study only on artificial networks. Why don’t you do the study on empirical networks instead (or in addition to) the artificial ones? I think the results would be much more convincing with real species interactions and their real body-masses rather than the ones derived from niche model and a few arbitrary heuristics, as you note it yourself (Appendix 1). It seems to me you wouldn’t have to change your methodology to derive the Lotka-Volterra parameters from the network structure. I guess your challenge is that you initially need a network at equilibrium to isolate the effect of forced extinction on the target species extinction time, but you could find an equilibrium state for empirical networks too, or adapt/remove networks which don’t have local stability.
  \end{quotation}


  \textbf{R:} We did not use empirical networks for two reasons, one of which the Reviewer touches upon. First and foremost, there are simply not many empirical networks with attached body mass data for all species. This limits the scope of network size and connectance that can be covered and, especially if the empirical networks were collected by different groups, introduces noise due to different field conditions, sampling strategies, etc. These issues would prevent us from drawing robust conclusions from empirical networks and are the main reason we simulated networks in the first place. While there are also limitations to simulation studies, we feel that the benefits of large sample size, large ranges of size and connectance, and consistent `sampling' methodology are important advantages in our case.
  Second, as the Reviewer notes, our study design requires a web that begins at equilibrium in order to isolate the effect of a species removal. Finding equilibrium states for empirical networks is not straightforward, especially across a large number of networks. On the other hand, removing networks not at equilibrium would diminish an already small sample. While we agree that empirical networks are interesting, the difficulty in finding a large set of sufficiently data-rich, equilibrium networks means that repeating our analyses with empirical networks is beyond the scope of the current study.
  We hope that as sampling methods continuously improve an empirically-based study may become feasible in the future.


  \begin{quotation}
  However, the overall methodology is quite complex, sometimes over-complex in my opinion. Even though I took time to understand it as fully as I could, I couldn’t fully understand the data analysis part, which is problematic as it prevented me to assess the adequation of the conclusions with the evidence. Regarding this statistical methods (e.g. PLS, predictor scaling, see my major comments), not only some points lack clarity, but they often lack a motivation compared to simpler and probably more suited alternatives. My main suggestion is to drastically simplify by removing the PLS, the simple linear regressions (l187 to l193) and you linear mixed models (l194 to 199), and replace all that by 3 multiple linear regressions over all the (6000)networks using time to extinction as the output variable and the following predictors: motif roles (count OR degree standardize OR Z-score), C, R, C:R, degree, STL. I think you would basically get all the answers you are seeking from these 3 models, or I missed something.
  \end{quotation}


  \textbf{R:} We agree that our methodology is quite complex. However, the proposed multiple linear regression will not suffice to answer our question of interest. We are primarily interested in how motif roles relate to time to extinction and secondarily interested in how motif roles relate to simpler measures (degree and trophic level). With respect to our first question: motif roles, as inherently multivariate descriptors, don't fit well in a linear regression together with univariate predictors. The univariate predictors may have larger effects than any motif individually, but this undervalues the effect of the motif role as a whole. The idea of simplifying a linear regression by removing some elements of a motif role also does not make much sense conceptually.

  Regarding our second question: we could indeed fit and interpret the suggested multiple linear regression. However, this would not tell us how motif roles vary with degree and STL. This is an essential but surprisingly neglected piece of background information that we feel makes an important contribution to efforts to interpret motif roles.
  We appreciate that this second goal was not clear in our previous draft and have revised the manuscript with this in mind.

  ** we could fit regressions for motifs + global, degree + global, and STL + global as a way to see whether motifs are better than simples?

  \begin{quotation}
  Regarding the form, I think some parts of the methods and results section may be passed to appendice or presented in a much simplified way. I would suggest condensing the three first paragraphs of Results section, which basically say the same things, into one much shorter. One option I would like would be to keep just one type of motif roles in the main manuscript results (the Z-score?), and describe the others in the appendice.
  \end{quotation}

  \textbf{R:} This is an excellent suggestion. Certainly the count can go in the appendices. Not sure about proportion vs. Z-score. Thoughts?


  \textbf{Major comments:}

    1. Are perturbation and disturbance interchangeable? [done]

      - l54: Now you use perturbation instead of disturbance: Are you talking about the same concept? You seem to use interchangeably those concepts in the following.

      \textbf{R:} Yes, we are talking about the same concept. We now use `disturbance' exclusively in order to avoid confusion.


    2. Motivation for using stable network unclear

      - l80: Why networks “networks at stable equilibria”? You don’t justify this methodological choice here, and neither in the methodology section. My guess is that you initially need a network at equilibrium to isolate the effect of the introduced extinction on the targeted species extinction time from a potential effect of an initial disequilibrium state. Whatever is you real reason, it is important to motivate it to justify the extra methodological complexification required to obtain this equilibrium state.

      \textbf{R:}
      Well he/she figured out why, but I guess we can state that explicitly


    3. Specify disturbance earlier

      - (l81) You only specify now (and not even fully) what was meant by disturbance/perturbation to the network: The removal of a species in the network. First, why using the term disturbance which is vague, while you could use extinction instead? Second, why do you chose this type of disturbance? It should be justified regarding the hypothesis you test. Plus, it doesnt appear consistent with the definitions of stability found in the corpus of articles from which you derive your hypotheses on stable motifs: Indeed, both the local stability (May, 1972) and the Quasi Sign stability (Allesina and Pascual, 2008, and later Borrelli, 2015) are related to the stability of populations to small perturbations around the equilibrium state. So why dont you define your disturbance as a small perturbation of species populations in your network instead of an extinction? I quickly read the cited articles and I may have missed some link with your work, but bringing a justification for this choice of disturbance, which determines time to extinction and is thus at the center of your methodology, is critically important.

      \textbf{R:}
      I guess "robustness" is actually the term usually used to talk about secondary extinctions following a primary extinction? Although I guess that's not about time to extinction but simply number of secondary extinctions. But it seems this reviewer doesn't really know the stability literature and doesn't know that local stability isn't something that we can just cause a small disturbance and then see how individual species respond. Maybe we should give a general "stability is a multifaceted concept that many people are interested in (refs). The aspect of stability we focus on is....".


    4. Controlling for trophic level not clear

      -(l77) This is unclear to me why you have to control for trophic level while testing the effect of motif roles on time to extinction? If the purpose is just to test if trophic level, which is indeed a simpler species level measure, is more suited to describe the species stability, then write it more clearly here.

      \textbf{R:}
      I thought the sentence before explained it pretty well, but maybe not.... It says "In addition, species with different75trophic levels tend to have different motif roles (Cirtwill & Ekl ̈of, 2018). Any test for a76relationship between motif roles and time to extinction should therefore take degree and77trophic level into account to avoid confounding these effects."


    5. Missing sentence?

      -l.85 to l.88: I dont get where does that sentence come from. Is that derived from earlier results? Then they should be referred here. Otherwise, this is a result of your study and should be in the results section.

      \textbf{R:}
      Oh this is a good point. This is in the abstract and we essentially say "these are our questions" and then say "Taken together, these85tests show that species are generally consistent in their vulnerability to disturbances,86regardless of the location in the network of that disturbance, and this vulnerability is87shaped by both motif roles and other network parameters." Maybe we could simply rephrase as "Taken together, the results we obtain for these tests show..."


    6. Vague sentence

      -l127: This last statement is a bit vague and misleading. Your network normalization will tell how the species stability is related with how much it participates to a given motif compared to the participation to this motif over all species of the community. The explanation provided l164 is much better.

      \textbf{R:}
      L127 says: "...while the network normalization indicates whether125trends in stability with participation in different motifs are related to how unusual each126species is within its community context." L164 says: " Third, to163understand whether it is the absolute frequency of motifs or the relative frequency164compared to other species in the network that is related to time to extinction" I think they say the same thing, but I perhaps the first sentence is a bit harder to understand for a non-native speaker. Maybe rephrase something like "...while the network normalization indicates whether trends in stability due to participation in different motifs are related to how unusual its motif participation is relative to other species in its community" Or is it more "how unusual within the community the motifs are that a species participates in"?


    7. Justify choice of threshold

      -l132: Why 1e10-5? Did you found this to be a threshold below which all species tend to vanish? On the contrary, how likely is it that the population grows again after passing under such threshold? Does it exist instead a grounded rule to determine a threshold per species depending on its bodymass and biomass? It would be much better, but I’m not familiar with Lotka-Volterra dynamics.

      \textbf{R:} We established this threshold because, like other simulation studies, we observed a few instances of populations declining to extremely low biomass and then rebounding. While such rebounds are rare, they are ecologically implausible and should be excluded. It is therefore common practice to set a threshold at which a population size is considered to be 0.
      
      
      We are not aware of any grounded rule or consensus threshold value to use in such simulations, however. For example, comparable simulation studies use thresholds of 1e10-12 (Binzer et al., 2012) and 1e10-20 (Ryser et al., 2019) without an explicit justification of these thresholds in the text. Our threshold, while admittedly arbitrary, is intermediate in this range of previous thresholds.
      
      
      To demonstrate whether the choice of threshold is likely to have influenced our results, we repeated a subset of our simulations using both of the thresholds mentioned above. Because we did not find network size or connectance to strongly influence time to extinction, and in the interest of completing this revision in a timely manner, we used only networks of 70 species and a connectance of 0.12 (intermediate size and connectance).
    %   I assume this was relatively arbitrary and I don't remember how often the population grows again after passing under the threshold, but it did happen right? (Which was why we had to introduce it?) We also talked about a threshold depending on bodymass didn't we? I now don't remember why we decided not to use it. (Also interesting that we got a reviewer who doesn't know the stability literature or lotka volterra - how many of those are there who are motif experts? Maybe many?

    %   - I don't remember how many came back ... it was arbitrary and I have no knowledge of an algorithm for getting thresholds based on body mass (don't see why biomass should matter). If there's a good way to extract body mass from the Julia code then maybe threshold could = 1 body mass?


    8. Rejection of PLS method and misunderstanding of study aim

      -l153-156:  I don’t think the PLS method is fitted to the purpose here and neither the simplest method to answer your question on quantifying the individual effect of each motif role/participation on the time to extinction. Regarding parsimony, as you have a “large” sample of networks, I don’t see a need for a latent space with reduced dimensionality. Regarding relevance, you wrote earlier (e.g. the previous sentence), and it’s meant by the coef you reconstruct later l.177, that you want here to estimate the effect of each motif role independently (question A), and now you just slide to another question that comes out of the blue: You want to get the effect of participation to a combination of motifs (question B) using PLS. But, if you want to answer question A, which I understood as your real goal, you can use a much simpler method: A multiple linear regression of the “time to extinction” (dependent variable) versus the motif roles, and your other variables (explanatory variables).

      \textbf{R:}
      The motif role is the 13-dimensional set of participation in all motifs. What we really want to understand is how the 'shape' of this vector relates to time to extinction. The PERMANOVA test tells us whether the motif role as a whole is related to time to extinction, but not which motifs might be particularly important. An equivalent question in community ecology would be `Is diversity different between sites and, if so, what groups drive this difference?` Our main goal is closer to question B, but as question A is simpler we answered it `on the way'.



    9. Description of STL not clear

      -l186: The description of STL in this sentence is unclear and not consistent with the description in the previous sentence, so you better remove this sentence.

      \textbf{R:}
      Actually this sentence is a continuation of the description of STL from the previous sentence: "shortest trophic level (STL): the length of the shortest food chain between the focal species184and any basal resource (Hairston & Hairston, 1993). Basal resources are assigned a trophic185level of one and other species are assigned an STL of one plus the STL of their prey." Maybe rephrase like "shortest trophic level (STL): calculated as the shortest food chain between the focal species and any basal resource (). To calculate STL, basal resources are assigned a trophic level of 1, and other species are assigned an STL of one than their lowest trophic level prey".


    10. Another complaint that PLS not fit for purpose and misunderstanding of aim

      -l187: Again, the chosen method doesn’t appear fit to the purpose. What you want to do is jointly measure the relative effects of the different predictors, i.e. the motif roles, degree and STL, on the time to extinction, which you can do with a multiple linear regression of the time to extinction (dependent variable Y) versus all predictors at once (explanatory). Actually, a multiple linear regression would answer two of your questions at once, namely the relative effect of each motif role compared to others AND the relative effect of motif roles VS other predictors, and simplify your methodology.
      
      \textbf{R:} No, we do not want to jointly measure the relative effects of the different predictors. We are not particularly interested in whether degree explains time to extinction better than participation in isolated motifs. Instead, we are specifically interested in how motif participation varies with degree and trophic level. Our aim here is to relate a species' motif role more explicitly to other network properties than has been done previously.

      Refers to this sentence "We then fit a series of linear regressions relating degree or STL and the count (taken187from the raw motif role), proportion (taken from the degree-normalized motif role), or188Z-score (taken from the network-normalized motif role) of each motif in the species’ role189using a series of linear regressions (78 regressions total; Table 1"


    11. Comment not clear [done]

      - l. If your predictors are scaled you should compare directly your the fitted coefficient values between predictors to assess which

      \textbf{R:} The line reference and part of this comment are missing. We apologize but cannot work out what this comment refers to in order to address it.


    12. Confusion about de-scaled effects

      -l225: I’m totally lost with this sentence. What do you mean by “effect” here? Is it the sum of coefficients of the predictor across PLS axes? Or is it its product with the standard deviation of the predictor across the dataset? Then, you tell that the motif roles (counts) effects appear small because the variables are scaled, but the fitted coefficients should be comparable between variables, no?  Then, you talk about the “un-scaled effects” and here I don’t know what you mean and I find myself unable to interpret for comparing STL/degree with motif roles in all PLS models.

      \textbf{R:}
      Referring to this sentence: "Although these224effects appear small when considering the scaled, centered predictors, the large ranges of225stable motifs which appear in species’ roles mean that the un-scaled effects of different226numbers of motifs are actually large (Fig. 2B)."

      - similar to R1 - seems to not understand that the un-scaled effects are per unit increase in predictor (TL of 4 vs 3, degree of 14 vs 15, participating in 1 vs. 2 chain motifs.)


  \textbf{Minor comments:}

    13. Meaning of motif roles not clear in abstract [done]

      l7  I’m unsure about meaning of a species motif roles at this stage. You define it later as the frequency with which it participates in different motifs (l64), but you should introduce it directly here or be more explicit.

      \textbf{R:} We have added a definition of motif roles to the abstract. To avoid confusion we have also retained the definition in the introduction.

      \begin{quotation}
       We test whether a species' time to extinction following a perturbation is related to its participation in stable and unstable motifs and assess how a species' motif role (the frequencies with which the species appears in each motif) co-varies with degree or trophic level.
      \end{quotation}


    14. Potential for larger motifs to affect stability [done?]

      L47-51: Another possible interpretation is that higher size motifs (or some whole network architectures), containing the three species motifs mentionned earlier, are determinant for stability. This would mean that the frequencies of the three species motifs themselves are not a driver but is induced by the frequency of higher size motifs. I think that determining if there exist some higher size motifs that explain networks stability better than the three species motifs is an open problem and would deserve discussion.

      \textbf{R:} The Reviewer is correct that motifs of different sizes are closely related. However, our understanding is that smaller motifs constrain larger ones. The number of one-way and two-way interactions (two-species motifs) constrain the number of three-way motifs that can occur and whether any of the two-way motifs can occur. Similarly, no four-species motif which includes two predators consuming a common prey can occur if the three-species direct competition motif does not occur. Larger motifs may indeed by a better predictor of stability than three-species motifs, but we are not aware of any studies of these larger motifs which could motivate such a discussion. If the Reviewer can suggest any, we would be happy to include them. In the meantime, we have added a line stating that we focus on three-species motifs since they are the most-commonly studied size of motif with respect to stability.


      \begin{quotation}
          In addition to providing species-level information on network structure, there are early indications that some meso-scale structures may tend to stabilize food webs~\citep{Prill2005,Borrelli2015,Monteiro2016}. 
          As three-species motifs are the best-studied in this context, we focus on this level of meso-scale structure.
      \end{quotation}



    15. Confusion about model relating extinction to STL

      -l194: In this paragraph, I’m totally lost: I don’t understand the question, nor how the models answer it. Are you trying to test whether the values of degrees/STL may affect the response of time to extinction to motif roles? The sentence must be clarified.

      \textbf{R:}
      Referring to: "We also wanted to estimate how the associations between motif roles and degree or STL194might affect the relationships between motif roles and mean time to extinction. To do this,195we fit two linear mixed-effect models relating mean time to extinction to a species’ degree196(or STL) and a random effect of the combination of network size and connectance. Both197models were fit using the function ‘lmer’ from the R (R Core Team, 2016) package198lmerTest(Kuznetsovaet al., 2017)."


    16. Small language edits [done]

      -l214: “a species” is repeated.

        \textbf{R:} Corrected

      -l300: I think “which” may be removed from this sentence.

        \textbf{R:} Changed `by which' to `whereby'

      -l306: Except for the omnivory motif, no?

        \textbf{R:} The omnivory motif was notably less stable than the apparent competition, direct competition, and three-species chain motifs (the most stable motifs), despite being much more stable than the 9 other three-species motifs. To avoid any confusion, we now parenthetically specify the three most stable motifs.


    17. Sarcasm...

      - l313: Which is typically what a multiple regression would disentangle…

      \textbf{R:} Line refers to: 
              The correlations between different measures of a species' role makes identifying the precise mechanisms affecting mean times to extinction difficult but, because each network measure provides some unique information, species with multiple `risk factors' (i.e., low degree, participation in many unstable motifs, and/or high trophic level) may be more vulnerable than species with only one of the above.


    18. Request for Lotka-Volterra equations [Kate?]

      Appendix l26: can you be more explicit on how you compute these metabolic rates? Further l46 you also say talk about a consumption rate. In this appendix, can you write the form of the Lotka-Volterra equations as a function of the body masses, the metabolic rates and the consumption rates? It would make the understanding .

      \textbf{R:}
      Fair, we just say they're "based on each species' body mass" - I assume it's a 3/4 power but I'll double check to remind myself.


    19. Suggestion to move line L111 earlier (to main text?)

      Appendix L111 : This sentence well condensates the question associated with your methodology, but I realise now it didn’t appear as clearly earlier, especially it lacks to the introduction section, you should put it earlier.


      \textbf{R:} The line he likes is:   To answer this question, we used a set of partial least squares (PLS) regressions to identify combinations of motifs which, together, explain substantial variation in time to extinction. 


    20. Caption of Table S1 unclear. [done?]

      - Table S1: The text is unclear: “Here we show the mean correlation among
      extinction orders across all removed species (R2) and all 100 simulated networks for each combination of S and C…“ → The mean correlation between extinction orders and what??


      \textbf{R:} As stated in Appendix lines 58-69, this is the mean of correlations of times extinction for each species across all removals. We have slighly revised the caption to make this clearer.


      \begin{quotation}
        Here we show the mean of correlations of extinction orders for a focal species across all removed species ($R^2$)
      \end{quotation}


\end{letter}

\end{document}


